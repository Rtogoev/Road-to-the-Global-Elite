\documentclass[12pt]{article}
\usepackage[utf8]{inputenc} 
\usepackage[english, russian]{babel}
\usepackage{amsmath}
\usepackage{amsfonts}
\linespread{1.3}
\begin{document}
\begin{titlepage}
\begin{center}
\textbf{\huge Теория  и решение примеров Шага 5, Ступени 1}
\end{center}
\end{titlepage}
\tableofcontents
\newpage
\section{\S Основные правила комбинаторики}

Теория отлично дана в книге, поэтому сюда я ее не переписывал.\\
Условия тоже не переписываются.

\newpage
\subsection{Задание 1}

Тут надо знать, что 000 для цифр быть не может\\
Способ решения является следствием из правила умножения. У нас есть 3 позиции одного типа(для цифр) и 3 позиции другого типа(для букв). Для первого типа количетво всех возможных значений равно 10, для второго - 12. В учебнике аналогичный пример, только количество позиций каждого типа равно 1. В любом случае, в таких ситуациях  количество всех возможных значений - это основание, а количество позиций - это степень.\\
Слеовательно, всех  вариантов с цифрами может быть:\\
$10^{3}-1=999$
\\
Для букв:\\
$12^{3}$
\\
Правильный ответ (по правилу умножения):\\
$12^{3}*999=1726272$

\newpage
\subsection{Задание 2}
Тут все просто, 4 позиции,  количество всех возможных значений 10.
$10^{4}=10000$

\newpage
\subsection{Задание 3}
Тут нужно понять, сколько видов бутеров у нас получается и составить решение по правилу умножения для каждого типа.\\
Первый тип, когда в бутере есть все компоненты.\\
Хлеб: 1 позиция, 3 вида хлеба = 3 в степени 1 = 3.\\
Колбаса: 5.\\
Масло: 1.\\
Количество всех возможных вариантов для первого типа бутеров:\\
 $3\cdot5\cdot1=15$
\\
\\
Второй тип, когда в бутере нет колбасы.\\
Хлеб:3.\\
Масло: 1.\\
Количество всех возможных вариантов для второго типа бутеров:\\
 $3\cdot1=3$
\\
\\
Третий тип, когда в бутере нет масла.\\
Хлеб:3.\\
Колбаса: 5.\\
Количество всех возможных вариантов для третьего типа бутеров:\\
 $3\cdot5=15$
\\
Для всех типов:\\
$15+15+3=33$

\newpage
\subsection{Задание 4}

От А до К, исключая Ё и Й будет 10 букв.\\
Цифр тоже 10. \\
1 позиция для букв, 3 для цифр:\\
$10 $(букв)$\cdot10$(цифр)$\cdot10$(цифр)$\cdot10$(цифр)$ = 10000 $

\newpage
\subsection{Задание 5}

Тут подвох в том, что правильных ответа 3. Ведь один и тот же человек может решить все хадачи(правило умножения), любые 4 человека могут быть выбраны из 20(порядок не важен - правило сочетаний) и каждая задача может быть предначертана преподом конкретному студенту(порядок важен - правило размещений).\\
Поэтому:\\
по правилу умножения:\\
$20^4$\\
по правилу сочетаний\\
$C_n^{k}  = \frac{20\cdot19\cdot18\cdot17}{1\cdot2\cdot3\cdot4} = 4845$\\
по правилу размещений\\
$A_n^{k}  = 20\cdot19\cdot18\cdot17 = 116280$

\newpage
\subsection{Задание 6}

n=36, k =3\\
Иногда проще решать задачу наоборот. Вытащим всех тузов из колоды - количетсво всех неинтересующих нас случаев:\\
 $C_{32}^{3}$\\
Количество вообще всех случаев:\\
$C_{36}^{3}$\\
\\
Тогда проще вычесть из всех неинтересующие случаи, тогда получим только интересющие!\\
$C_{36}^{3} -  C_{32}^{3}$\\

\newpage
\subsection{Задание 7}

$C_{10}^{3}$\\

\newpage
\subsection{Задание 8}

a) 16!, потому что нужно составить все возможные варианты очередей(правило перестановок)\\
б) $A_{16}^{3}$

\newpage
\subsection{Задание 9}

$n=2^6=64$\\
Исключаем вариант "все решки" и все варианты "1 орла": 64-1-6=57\\

\newpage
\subsection{Задание 10}

$n_1=20$\\
$n_2=3$\\
$C_{20}^5\cdot3$

\newpage
\section{\S Случайное событие. Вероятностное пространство. Классическое определение вероятности.}

\subsection{Задание 11}

1)например, 6,6, орел.\\
2)6*6*2=72\\
3)дублей с орлом всего может быть 6, тогда\\
$p($дубль с орлом$) = \frac{6}{72}$

\newpage
\subsection{Задание 12}

позиций = 4, алфавит = 2, тогда всего исходов:\\
$2^4=16$\\
Количество исходов, когда нет орлов = 1.\\
Есть хотя бы 1 орел:$16 - 1 = 15$\\
$p($хотя бы 1 орел$) = \frac{15}{16}$

\newpage
\subsection{Задание 13}

позиций = 2, алфавит = 6\\
Всего: $6^2=36$\\
интересующие нас случаи(их 5):\\
 2-6,\\
 3-5,\\ 
4-4,\\
 5-3,\\
 6-2\\
$p($сумма очков равна 8$)=\frac{5}{36}$

\newpage
\subsection{Задание 14}

позиций = 3, алфавит = 6.\\
Всего исходов: $6^3=216$\\
Нас интересуют случаи(их 4):\\
666\\
665\\
656\\
566\\
\\
$p($сумма очков больше 16$)=\frac{4}{216}$

\newpage
\subsection{Задание 15}

позиций = 5, алфавит = 6.\\
Всего: $6^5$\\
Нас интересуют случаи(их 6):\\
11111\\
11112\\
11121\\
11211\\
12111\\
21111\\
$p($сумма мегьше, либо равна 6$) =\frac{6}{6^5}=\frac{1}{6^4}$ 

\newpage
\subsection{Задание 16}

позиций = 2, алфавит = 6\\
Всего: $6^6=36$\\
Нас интересуют:\\
6-1\\
6-2\\
6-3\\
6-4\\
6-5\\
\\
1-6\\
2-6\\
3-6\\
4-6\\
5-6\\
\\
$p($не более одного раза$)=\frac{10}{36}=\frac{5}{18}$

\newpage
\subsection{Задание 17}


позиций = 4, алфавит = 10\\
Всего: $10^4=10000$\\
3 попытки. Тут странно, так как если ты ввел какой-нибудь пин-код, а он неверный, то вводить его еще раз ты не будешь. Значит, каждая следующая попытка уменьшает количество пинковод на 1, тем самым чуть-чуть увеличивая вероятность успеха. То есть\\
\\
$p($угадать пин-код с 3 попытки$)=\frac{1}{10000}+\frac{1}{9999}+\frac{1}{9998}$\\
Но в ответах почему-то $\frac{3}{10000}$

\newpage
\subsection{Задание 18}

$\frac{n}{k}$

\newpage
\subsection{Задание 19}

к сожалению, я не знаю, как это решить. Мне кажется, что в  условии чего-то не хватает.

\newpage
\subsection{Задание 20}

6 юношей, 14 девушек.\\
количество всех возможных способов вырать 2 юношей из 6:\\
$C_6^2=\frac{6\cdot5}{1\cdot2}=15$\\
количество всех возможных способов вырать 1 девушку из 14:\\
$C_{14}^1=14$\\
колиество способов выбрать 3 любых студента из всех(6+14=20):\\
$C_{20}^3=\frac{20\cdot19\cdot18}{1\cdot2\cdot3}=20\cdot19\cdot3$\\
\\
$p=\frac{C_6^2\cdot C_{14}^1}{C_{20}^3}=\frac{14\cdot15}{20\cdot19\cdot3}=\frac{7}{38}$

\newpage
\subsection{Задание 21}

количество всех возможных способов вырать 3 из 12:\\
$C_{12}^3=\frac{12\cdot11\cdot10}{1\cdot2\cdot3}=220$\\
колиество способов выбрать 3 любых из всех(12+3=15):\\
$C_{15}^3=\frac{15\cdot14\cdot13}{1\cdot2\cdot3}=455$\\
$C_{15}^3-C_{12}^3=455-220=235$\\
$p=\frac{C_{15}^3-C_{12}^3}{C_{15}^3}=\frac{235}{455}=\frac{47}{91}$

\newpage
\subsection{Задание 22}

$C_n^m$\\
В подобных задачах лучше чтобы у всех С, n было минимально. Тогда легче счистать.\\
Число интересующих исходов:\\
$C_{20}^3-(C_{5}^2\cdot C_{15}^1+C_{5}^3)$\\
$C_{5}^2=\frac{5\cdot4}{1\cdot2}=10$\\
$C_{15}^1=15$\\
$C_{5}^2\cdot C_{15}^1=150$\\
$C_{5}^3=\frac{5\cdot4\cdot3}{1\cdot2\cdot3}=10$\\
$C_{5}^2\cdot C_{15}^1+C_{5}^3=150+10=160$\\
$C_{20}^3=\frac{20\cdot19\cdot18}{1\cdot2\cdot3}=1140$\\
$C_{20}^3-(C_{5}^2\cdot C_{15}^1+C_{5}^3)=1140-160=980$\\
$p=\frac{C_{20}^3-(C_{5}^2\cdot C_{15}^1+C_{5}^3)}{C_{20}^3}=\frac{890}{1140}=\frac{49}{57}$

\newpage
\subsection{Задание 23}

Здесь проще наоборот, решаем случай, когда вообще нет юношей.
Это когда есть только девушки)\\
Число всех интересующий исходов в таком случае:\\
$C_{25}^3-C_{15}^3$\\
$p=\frac{C_{25}^3-C_{15}^3}{C_{25}^3}$\\
$C_{25}^3=2300$\\
$C_{15}^3=455$\\
$C_{25}^3-C_{15}^3=2300-455=1845$\\
$p=\frac{C_{25}^3-C_{15}^3}{C_{25}^3}=\frac{1845}{2300}=\frac{369}{460}$

\newpage
\subsection{Задание 24}

На интересуют случаи, когда выбраны только 4 парня или когда выбраны 3 парня и 1 девушка:\\
$C_{10}^4+C_{10}^3\cdot C_{5}^1$\\
Тогда вероятность всех этих исходов будет:\\
$
p=\frac{C_{10}^4+C_{10}^3\cdot C_{5}^1}{C_{15}^4}
=\frac{810}{1365}=\frac{54}{91}
$


\newpage
\subsection{Задание 25}

Нас интересуют случаи, когда повезло 2 новичкам и одному бывалому и 3 новичкам:\\
$
p=\frac{C_{6}^3+C_{6}^2\cdot C_{9}^1}{C_{15}^3}
=\frac{135+20}{455}=\frac{31}{91}
$


\newpage
\subsection{Задание 26}
Хотя бы один, это значит 1 и более.\\
Проше решать обратную задачу - найти количество всех вариантов англоговорящих делегаций, далее из вообще всех вариантов вычесть это число. Получим как раз те случаи, когда в делегации есть хоть один неговорящий.
Число вариантов хорошо говорящих делегаций:\\
$
C_{6}^3
$\\
Число всех:\\
$
C_{10}^3
$\\
Число вариантов вообще не говорящих по английски делегаций:\\
$
C_{10}^3-C_{6}^3
$\\
Вероятность того, что в делегацию попадет хотя бы один неговорящий:\\
$
p=\frac{C_{10}^3-C_{6}^3}{C_{10}^3}=
\frac{120-20}{120}=\frac{5}{6}
$

\newpage
\subsection{Задание 27}

Нас интересуют случаи, когда проконтроллированы 2 брака и 2 нормальных трубы, и проконтроллированы все 3 брака и 1 нормальная труба:\\
$
p=\frac{C_{3}^2\cdot C_{12}^2+C_{3}^3\cdot C_{12}^1}{C_{15}^4}=
\frac{198+12}{1365}=
\frac{2}{13}
$

\newpage
\subsection{Задание 28}

$
p=\frac{C_{12}^3\cdot C_{10}^1+C_{12}^4}{C_{22}^4}=
=\frac{7}{19}
$

\newpage
\subsection{Задание 29}

1)
Тут проще сначала решать наоборот.\\
$
p=\frac{C_{23}^5-(C_{8}^1\cdot C_{15}^4+C_{15}^5)}{C_{23}^5}
$\\
2)
$
p=\frac{C_{15}^3 \cdot C_{8}^2}{C_{23}^3}
$

\newpage
\subsection{Задание 30}
Нужно найти вероятности прохождения первого и второго туров.\\
$
p_1=\frac{C_{25}^3\cdot C_{5}^1+C_{25}^4}{C_{30}^4}\\
p_2=\frac{C_{18}^3\cdot C_{6}^1+C_{18}^4}{C_{24}^4}
$\\
Тут придется сначала прочитать теорию к следующе главе, чтобы знать, почему вероятности исходов первого и второго тура в коннце надо умножить.\\
$
p_1\cdot p_2=\frac{C_{25}^3\cdot C_{5}^1+C_{25}^4}{C_{30}^4}\cdot \frac{C_{18}^3\cdot C_{6}^1+C_{18}^4}{C_{24}^4}
$

\newpage
\section{\S Операции с событиями, формула сложения вероятностей, независимые события}

Чтобы здесь хоть что-то решить, лучше полностью выучить теорию из всех прерыдущих глав.

\newpage
\subsection{Задание 31}

Тут ошибка в ответах!

n=36\\
A - на 1 кости четное\\
B - на 1 и 2 кости в сумме больше 3\\
Число исходов события В проще посчитать, если посчитать число исходов обратных В и вычесть это число из всех.
Всего исходов для $\overline{B}$:\\
11\\
12\\
21\\
Тогда,\\
$
n_B=36-3=33\\
n_A=3\cdot6=18\\
P(A)=\frac{18}{36}=\frac{1}{2}\\
P(B)=\frac{33}{36}=\frac{11}{12}\\
$\\
а)
$
A \cap B:\\
22,
23,
24,
25,
26,
\\
41,
42,
43,
44,
45,
46,
\\
61,
62,
63,
64,
65,
66.
\\
n_{A \cap B}=6+6+5=17\\
P(A \cap B) = \frac{17}{36}\\
$
б)
$
P(A \cup B) = P(A)+P(B)-P(A \cap B)=\frac{1}{2}+\frac{11}{12}-\frac{17}{36}=\frac{34}{36}=\frac{17}{18}
$\\
в)
$
P(A)=\frac{18}{36}=\frac{1}{2}
$\\
г)
$
P(\bar{A})=\frac{18}{36}=\frac{1}{2}
$\\
д)
$
n_{\overline{A \cap B}}=36-17=19
P(\overline{A \cap B})=\frac{19}{36}
$


\newpage
\subsection{Задание 32}

A - Анжи победит МЮ\\
B - Зенит победит Барселону\\
С -  наши победят\\
D - только одна наша команда победит\\ 
E - никто из наших не победит\\
F - выиграет только Зенит\\
$\\
P(A)=0.3\\
P(B)=0.4\\
P(C)=P(A \cap B)=0.3\cdot 0.4=0.12\\
P(D)=P(
(A \cap \overline{B})
 \cup
 (\overline{A} \cap {B})
)=P(A)\cdot P(\overline{B})+P(\overline{A})\cdot P( B)
=0.3\cdot 0.6+0.7\cdot 0.4=0.76\\
P(E)=P(\overline{A}\cap \overline{B})=P(\overline{A})\cdot P(\overline{B})=0.6\cdot 0.7=0.42\\
P(F)=P(\overline{A}\cap B)=0.28
$

\newpage
\subsection{Задание 33}

n=36\\
А - дубль\\
В - в сумме больше 9\\
$
P(A\cup B) - ?\\
n_A=6\\
n_B=6\\
A\cap B: 6-6, 5-5.\\
n_{A\cap B}=2\\
P(A)=\frac{6}{36}\\
P(B)=\frac{6}{36}\\
P(A\cap B)=\frac{2}{36}\\
P(A\cup B)=P(A)+P(B)-P(A\cap B)=
\frac{6}{36}+\frac{6}{36}-\frac{2}{36}=
\frac{10}{36}=\frac{5}{18}
$

\newpage
\subsection{Задание 34}

$
P(A\cap B)=P(A)\cdot P(B)\\
P(A)=0.4\\
P(B)=0.9\\
P(
(A \cap \overline B)
\cup 
(\overline  A\cap B)
\cup 
P(\overline B)\cdot P(\overline A)
)=P(A\cap \overline B)+
P(\overline A\cap B)+ 
P(\overline B)\cdot P(\overline A)=
P(A)\cdot P(\overline B) +  
P(\overline A)\cdot P( B)+
P(\overline A)\cdot P(\overline B)=
0.4\cdot 0.1+0.6\cdot 0.9+0.1\cdot 0.6=0.64 
$

\newpage
\subsection{Задание 35}

А - книга есть в первой библиотеке
В - книга есть во второй библиотеке
$
P(A)=0.7\\
P(B)=0.5\\
P(
(A\cap \overline B)\cup 
(\overline A\cap B)\cup 
(A\cap B)
)=
P(A)\cdot P(\overline B)+
P(\overline A)\cdot P(B)+
P(A)\cdot P(B)=
0.7\cdot 0.5+0.3\cdot 0.5 + 0.7\cdot 0.5 = 0.5\cdot(0.7+0.3+0.7)=
0.5\cdot 1.7=0.85
$

\newpage
\subsection{Задание 36}

$
P(A)=0.4\\
P(B)=0.7\\
P((A\cap \overline B)\cup (\overline A \cap B))=
P(A)\cdot P(\overline B)+
P(\overline A)\cdot P(B)=
0.4\cdot 0.3+0.6\cdot 0.7=0.54
$

\newpage
\subsection{Задание 37}

$
P(A)=0.6\\
P(B)=0.4\\
P((A\cap \overline B)\cup 
(\overline A \cap B)\cup 
(A\cap B))=
P(A)\cdot P(\overline B)+
P(\overline A)\cdot P(B)+
P(A)\cdot P(B)=
0.4\cdot 0.4+0.6\cdot 0.6 + 0.6 \cdot 0.4=0.76
$

\newpage
\subsection{Задание 38}

A - первый студент придет в срок\\
В - второй студент придет в срок\\
$
P(A)=0.8\\
P(B)=0.7\\
P((A\cap \overline B)\cup 
(\overline A \cap B))=
P(A)\cdot P(\overline B)+
P(\overline A)\cdot P(B)=
0.2\cdot 0.7+0.8\cdot 0.3=0.38
$

\newpage
\subsection{Задание 39}

А - увидеть на телевидении\\
В - прочитать в прессе\\
$
P(A)=0.7\\
P(B)=0.4\\
P(A\cap \overline B)=
0.7\cdot 0.6=0.42\\
P(A\cup B)=P(A)+P(B)-P(A\cap B)=
0.7+0.4-0.7\cdot 0.4=0.82
$

\newpage
\subsection{Задание 40}

А - отлично по первому предмету\\
В - отлично по второму предмету\\
$
P(A)=0.3\\
P(B)=0.5\\
P(A\cap B)=P(A)\cdot P(B)=0.3\cdot 0.5=0.15\\
P((A\cap \overline B)\cup 
(\overline A \cap B))=
P(A)\cdot P(\overline B)+
P(\overline A)\cdot P(B)=
0.3\cdot 0.5+0.7\cdot 0.5=0.5
$

\newpage
\subsection{Задание 41}

А - первый студент опоздает\\
В - второй студент опоздает\\
$
P(A)=0.2\\
P(B)=0.6\\
P((A\cap \overline B)\cup 
(\overline A \cap B)\cup 
(A\cap B))=
P(A)\cdot P(\overline B)+
P(\overline A)\cdot P(B)+
P(A)\cdot P(B)=
0.8\cdot 0.6+0.2\cdot 0.4 + 0.2 \cdot 0.6=0.68
$

\newpage
\section{\S Условная вероятность}

\subsection{Задание 42}

A - сумма очков бюольше 8\\
В - выпало четное число\\
$
n=36\\
n_{A\cap B}=3\\
n_B=9\\
P(B)=\frac{9}{36}\\
P(A\cap B)=\frac{3}{36}\\
p(A|B)=\frac{P(A\cap B)}{P(B)}=\frac{3}{36} : \frac{9}{36}=\frac{1}{3}
$
\end{document}