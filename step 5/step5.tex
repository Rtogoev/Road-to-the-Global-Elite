\documentclass[12pt]{article}
\usepackage[utf8]{inputenc} 
\usepackage[english, russian]{babel}
\usepackage{amsmath}
\usepackage{amsfonts}
\linespread{1.3}
\begin{document}
\begin{titlepage}
\begin{center}
\textbf{\huge Теория  и решение примеров Шага 5, Ступени 1}
\end{center}
\end{titlepage}
\tableofcontents
\newpage
\section{\S Основные правила комбинаторики}

Теория отлично дана в книге, поэтому сюда я ее не переписывал.\\
Условия тоже не переписываются.

\newpage
\subsection{Задание 1}

Тут надо знать, что 000 для цифр быть не может\\
Способ решения является следствием из правила умножения. У нас есть 3 позиции одного типа(для цифр) и 3 позиции другого типа(для букв). Для первого типа количетво всех возможных значений равно 10, для второго - 12. В учебнике аналогичный пример, только количество позиций каждого типа равно 1. В любом случае, в таких ситуациях  количество всех возможных значений - это основание, а количество позиций - это степень.\\
Слеовательно, всех  вариантов с цифрами может быть:\\
$10^{3}-1=999$
\\
Для букв:\\
$12^{3}$
\\
Правильный ответ (по правилу умножения):\\
$12^{3}*999=1726272$

\newpage
\subsection{Задание 2}
Тут все просто, 4 позиции,  количество всех возможных значений 10.
$10^{4}=10000$

\newpage
\subsection{Задание 3}
Тут нужно понять, сколько видов бутеров у нас получается и составить решение по правилу умножения для каждого типа.\\
Первый тип, когда в бутере есть все компоненты.\\
Хлеб: 1 позиция, 3 вида хлеба = 3 в степени 1 = 3.\\
Колбаса: 5.\\
Масло: 1.\\
Количество всех возможных вариантов для первого типа бутеров:\\
 $3\cdot5\cdot1=15$
\\
\\
Второй тип, когда в бутере нет колбасы.\\
Хлеб:3.\\
Масло: 1.\\
Количество всех возможных вариантов для второго типа бутеров:\\
 $3\cdot1=3$
\\
\\
Третий тип, когда в бутере нет масла.\\
Хлеб:3.\\
Колбаса: 5.\\
Количество всех возможных вариантов для третьего типа бутеров:\\
 $3\cdot5=15$
\\
Для всех типов:\\
$15+15+3=33$

\newpage
\subsection{Задание 4}

От А до К, исключая Ё и Й будет 10 букв.\\
Цифр тоже 10. \\
1 позиция для букв, 3 для цифр:\\
$10 $(букв)$\cdot10$(цифр)$\cdot10$(цифр)$\cdot10$(цифр)$ = 10000 $

\newpage
\subsection{Задание 5}

Тут подвох в том, что правильных ответа 3. Ведь один и тот же человек может решить все хадачи(правило умножения), любые 4 человека могут быть выбраны из 20(порядок не важен - правило сочетаний) и каждая задача может быть предначертана преподом конкретному студенту(порядок важен - правило размещений).\\
Поэтому:\\
по правилу умножения:\\
$20^4$\\
по правилу сочетаний\\
$C_n^{k}  = \frac{20\cdot19\cdot18\cdot17}{1\cdot2\cdot3\cdot4} = 4845$\\
по правилу размещений\\
$A_n^{k}  = 20\cdot19\cdot18\cdot17 = 116280$

\newpage
\subsection{Задание 6}

n=36, k =3

Иногда проще решать задачу наоборот. Вытащим всех тузов из колоды - количетсво всех неинтересующих нас случаев:\\
 $C_{32}^{3}$\\
Количество вообще всех случаев:\\
$C_{36}^{3}$\\
\\
Тогда проще вычесть из всех неинтересующие случаи, тогда получим только интересющие!\\
$C_{36}^{3} -  C_{32}^{3}$\\

\newpage
\subsection{Задание 7}

$C_{10}^{3}$\\

\newpage
\subsection{Задание 8}

a) 16!, потому что нужно составить все возможные варианты очередей(правило перестановок)\\
б) $A_{16}^{3}$

\newpage
\subsection{Задание 9}

$n=2^6=64$\\
Исключаем вариант "все решки" и все варианты "1 орла": 64-1-6=57\\

\newpage
\subsection{Задание 10}

$n_1=20$\\
$n_2=3$\\
$C_{20}^5\cdot3$





\end{document}