\documentclass[12pt]{article}
\usepackage[utf8]{inputenc} 
\usepackage[english, russian]{babel}
\usepackage{amsmath}
\usepackage{amsfonts}
\linespread{1.3}
\begin{document}
\begin{titlepage}
\begin{center}
\textbf{\huge Теория вероятности 12}
\end{center}
\end{titlepage}
\tableofcontents
\newpage
\section{\S Основные правила комбинаторики}

Теория отлично дана в книге, поэтому сюда я ее не переписывал.\\
Условия тоже не переписываются.

\newpage
\subsection{Задание 1}

Тут надо знать, что 000 для цифр быть не может\\
Способ решения является следствием из правила умножения. У нас есть 3 позиции одного типа(для цифр) и 3 позиции другого типа(для букв). Для первого типа количетво всех возможных значений равно 10, для второго - 12. В учебнике аналогичный пример, только количество позиций каждого типа равно 1. В любом случае, в таких ситуациях  количество всех возможных значений - это основание, а количество позиций - это степень.\\
Слеовательно, всех  вариантов с цифрами может быть:\\
$10^{3}-1=999$
\\
Для букв:\\
$12^{3}$
\\
Правильный ответ (по правилу умножения):\\
$12^{3}*999=1726272$

\newpage
\subsection{Задание 2}
Тут все просто, 4 позиции,  количество всех возможных значений 10.
$10^{4}=10000$

\newpage
\subsection{Задание 3}
Тут нужно понять, сколько видов бутеров у нас получается и составить решение по правилу умножения для каждого типа.\\
Первый тип, когда в бутере есть все компоненты.\\
Хлеб: 1 позиция, 3 вида хлеба = 3 в степени 1 = 3.\\
Колбаса: 5.\\
Масло: 1.\\
Количество всех возможных вариантов для первого типа бутеров:\\
 $3\cdot5\cdot1=15$
\\
\\
Второй тип, когда в бутере нет колбасы.\\
Хлеб:3.\\
Масло: 1.\\
Количество всех возможных вариантов для второго типа бутеров:\\
 $3\cdot1=3$
\\
\\
Третий тип, когда в бутере нет масла.\\
Хлеб:3.\\
Колбаса: 5.\\
Количество всех возможных вариантов для третьего типа бутеров:\\
 $3\cdot5=15$
\\
Для всех типов:\\
$15+15+3=33$

\newpage
\subsection{Задание 4}

От А до К, исключая Ё и Й будет 10 букв.\\
Цифр тоже 10. \\
1 позиция для букв, 3 для цифр:\\
$10 $(букв)$\cdot10$(цифр)$\cdot10$(цифр)$\cdot10$(цифр)$ = 10000 $

\newpage
\subsection{Задание 5}

Тут подвох в том, что правильных ответа 3. Ведь один и тот же человек может решить все хадачи(правило умножения), любые 4 человека могут быть выбраны из 20(порядок не важен - правило сочетаний) и каждая задача может быть предначертана преподом конкретному студенту(порядок важен - правило размещений).\\
Поэтому:\\
по правилу умножения:\\
$20^4$\\
по правилу сочетаний\\
$C_n^{k}  = \frac{20\cdot19\cdot18\cdot17}{1\cdot2\cdot3\cdot4} = 4845$\\
по правилу размещений\\
$A_n^{k}  = 20\cdot19\cdot18\cdot17 = 116280$

\newpage
\subsection{Задание 6}

n=36, k =3\\
Иногда проще решать задачу наоборот. Вытащим всех тузов из колоды - количетсво всех неинтересующих нас случаев:\\
 $C_{32}^{3}$\\
Количество вообще всех случаев:\\
$C_{36}^{3}$\\
\\
Тогда проще вычесть из всех неинтересующие случаи, тогда получим только интересющие!\\
$C_{36}^{3} -  C_{32}^{3}$\\

\newpage
\subsection{Задание 7}

$C_{10}^{3}$\\

\newpage
\subsection{Задание 8}

a) 16!, потому что нужно составить все возможные варианты очередей(правило перестановок)\\
б) $A_{16}^{3}$

\newpage
\subsection{Задание 9}

$n=2^6=64$\\
Исключаем вариант "все решки" и все варианты "1 орла": 64-1-6=57\\

\newpage
\subsection{Задание 10}

$n_1=20$\\
$n_2=3$\\
$C_{20}^5\cdot3$

\newpage
\section{\S Случайное событие. Вероятностное пространство. Классическое определение вероятности.}

\subsection{Задание 11}

1)например, 6,6, орел.\\
2)6*6*2=72\\
3)дублей с орлом всего может быть 6, тогда\\
$p($дубль с орлом$) = \frac{6}{72}$

\newpage
\subsection{Задание 12}

позиций = 4, алфавит = 2, тогда всего исходов:\\
$2^4=16$\\
Количество исходов, когда нет орлов = 1.\\
Есть хотя бы 1 орел:$16 - 1 = 15$\\
$p($хотя бы 1 орел$) = \frac{15}{16}$

\newpage
\subsection{Задание 13}

позиций = 2, алфавит = 6\\
Всего: $6^2=36$\\
интересующие нас случаи(их 5):\\
 2-6,\\
 3-5,\\ 
4-4,\\
 5-3,\\
 6-2\\
$p($сумма очков равна 8$)=\frac{5}{36}$

\newpage
\subsection{Задание 14}

позиций = 3, алфавит = 6.\\
Всего исходов: $6^3=216$\\
Нас интересуют случаи(их 4):\\
666\\
665\\
656\\
566\\
\\
$p($сумма очков больше 16$)=\frac{4}{216}$

\newpage
\subsection{Задание 15}

позиций = 5, алфавит = 6.\\
Всего: $6^5$\\
Нас интересуют случаи(их 6):\\
11111\\
11112\\
11121\\
11211\\
12111\\
21111\\
$p($сумма мегьше, либо равна 6$) =\frac{6}{6^5}=\frac{1}{6^4}$ 

\newpage
\subsection{Задание 16}

позиций = 2, алфавит = 6\\
Всего: $6^6=36$\\
Нас интересуют:\\
6-1\\
6-2\\
6-3\\
6-4\\
6-5\\
\\
1-6\\
2-6\\
3-6\\
4-6\\
5-6\\
\\
$p($не более одного раза$)=\frac{10}{36}=\frac{5}{18}$

\newpage
\subsection{Задание 17}


позиций = 4, алфавит = 10\\
Всего: $10^4=10000$\\
3 попытки. Тут странно, так как если ты ввел какой-нибудь пин-код, а он неверный, то вводить его еще раз ты не будешь. Значит, каждая следующая попытка уменьшает количество пинковод на 1, тем самым чуть-чуть увеличивая вероятность успеха. То есть\\
\\
$p($угадать пин-код с 3 попытки$)=\frac{1}{10000}+\frac{1}{9999}+\frac{1}{9998}$\\
Но в ответах почему-то $\frac{3}{10000}$

\newpage
\subsection{Задание 18}

$\frac{n}{k}$

\newpage
\subsection{Задание 19}

к сожалению, я не знаю, как это решить. Мне кажется, что в  условии чего-то не хватает.

\newpage
\subsection{Задание 20}

6 юношей, 14 девушек.\\
количество всех возможных способов вырать 2 юношей из 6:\\
$C_6^2=\frac{6\cdot5}{1\cdot2}=15$\\
количество всех возможных способов вырать 1 девушку из 14:\\
$C_{14}^1=14$\\
колиество способов выбрать 3 любых студента из всех(6+14=20):\\
$C_{20}^3=\frac{20\cdot19\cdot18}{1\cdot2\cdot3}=20\cdot19\cdot3$\\
\\
$p=\frac{C_6^2\cdot C_{14}^1}{C_{20}^3}=\frac{14\cdot15}{20\cdot19\cdot3}=\frac{7}{38}$

\newpage
\subsection{Задание 21}

количество всех возможных способов вырать 3 из 12:\\
$C_{12}^3=\frac{12\cdot11\cdot10}{1\cdot2\cdot3}=220$\\
колиество способов выбрать 3 любых из всех(12+3=15):\\
$C_{15}^3=\frac{15\cdot14\cdot13}{1\cdot2\cdot3}=455$\\
$C_{15}^3-C_{12}^3=455-220=235$\\
$p=\frac{C_{15}^3-C_{12}^3}{C_{15}^3}=\frac{235}{455}=\frac{47}{91}$

\newpage
\subsection{Задание 22}

$C_n^m$\\
В подобных задачах лучше чтобы у всех С, n было минимально. Тогда легче счистать.\\
Число интересующих исходов:\\
$C_{20}^3-(C_{5}^2\cdot C_{15}^1+C_{5}^3)$\\
$C_{5}^2=\frac{5\cdot4}{1\cdot2}=10$\\
$C_{15}^1=15$\\
$C_{5}^2\cdot C_{15}^1=150$\\
$C_{5}^3=\frac{5\cdot4\cdot3}{1\cdot2\cdot3}=10$\\
$C_{5}^2\cdot C_{15}^1+C_{5}^3=150+10=160$\\
$C_{20}^3=\frac{20\cdot19\cdot18}{1\cdot2\cdot3}=1140$\\
$C_{20}^3-(C_{5}^2\cdot C_{15}^1+C_{5}^3)=1140-160=980$\\
$p=\frac{C_{20}^3-(C_{5}^2\cdot C_{15}^1+C_{5}^3)}{C_{20}^3}=\frac{890}{1140}=\frac{49}{57}$

\newpage
\subsection{Задание 23}

Здесь проще наоборот, решаем случай, когда вообще нет юношей.
Это когда есть только девушки)\\
Число всех интересующий исходов в таком случае:\\
$C_{25}^3-C_{15}^3$\\
$p=\frac{C_{25}^3-C_{15}^3}{C_{25}^3}$\\
$C_{25}^3=2300$\\
$C_{15}^3=455$\\
$C_{25}^3-C_{15}^3=2300-455=1845$\\
$p=\frac{C_{25}^3-C_{15}^3}{C_{25}^3}=\frac{1845}{2300}=\frac{369}{460}$

\newpage
\subsection{Задание 24}

На интересуют случаи, когда выбраны только 4 парня или когда выбраны 3 парня и 1 девушка:\\
$C_{10}^4+C_{10}^3\cdot C_{5}^1$\\
Тогда вероятность всех этих исходов будет:\\
$
p=\frac{C_{10}^4+C_{10}^3\cdot C_{5}^1}{C_{15}^4}
=\frac{810}{1365}=\frac{54}{91}
$


\newpage
\subsection{Задание 25}

Нас интересуют случаи, когда повезло 2 новичкам и одному бывалому и 3 новичкам:\\
$
p=\frac{C_{6}^3+C_{6}^2\cdot C_{9}^1}{C_{15}^3}
=\frac{135+20}{455}=\frac{31}{91}
$


\newpage
\subsection{Задание 26}
Хотя бы один, это значит 1 и более.\\
Проше решать обратную задачу - найти количество всех вариантов англоговорящих делегаций, далее из вообще всех вариантов вычесть это число. Получим как раз те случаи, когда в делегации есть хоть один неговорящий.
Число вариантов хорошо говорящих делегаций:\\
$
C_{6}^3
$\\
Число всех:\\
$
C_{10}^3
$\\
Число вариантов вообще не говорящих по английски делегаций:\\
$
C_{10}^3-C_{6}^3
$\\
Вероятность того, что в делегацию попадет хотя бы один неговорящий:\\
$
p=\frac{C_{10}^3-C_{6}^3}{C_{10}^3}=
\frac{120-20}{120}=\frac{5}{6}
$

\newpage
\subsection{Задание 27}

Нас интересуют случаи, когда проконтроллированы 2 брака и 2 нормальных трубы, и проконтроллированы все 3 брака и 1 нормальная труба:\\
$
p=\frac{C_{3}^2\cdot C_{12}^2+C_{3}^3\cdot C_{12}^1}{C_{15}^4}=
\frac{198+12}{1365}=
\frac{2}{13}
$

\newpage
\subsection{Задание 28}

$
p=\frac{C_{12}^3\cdot C_{10}^1+C_{12}^4}{C_{22}^4}=
=\frac{7}{19}
$

\newpage
\subsection{Задание 29}

1)
Тут проще сначала решать наоборот.\\
$
p=\frac{C_{23}^5-(C_{8}^1\cdot C_{15}^4+C_{15}^5)}{C_{23}^5}
$\\
2)
$
p=\frac{C_{15}^3 \cdot C_{8}^2}{C_{23}^3}
$

\newpage
\subsection{Задание 30}
Нужно найти вероятности прохождения первого и второго туров.\\
$
p_1=\frac{C_{25}^3\cdot C_{5}^1+C_{25}^4}{C_{30}^4}\\
p_2=\frac{C_{18}^3\cdot C_{6}^1+C_{18}^4}{C_{24}^4}
$\\
Тут придется сначала прочитать теорию к следующе главе, чтобы знать, почему вероятности исходов первого и второго тура в коннце надо умножить.\\
$
p_1\cdot p_2=\frac{C_{25}^3\cdot C_{5}^1+C_{25}^4}{C_{30}^4}\cdot \frac{C_{18}^3\cdot C_{6}^1+C_{18}^4}{C_{24}^4}
$

\newpage
\section{\S Операции с событиями, формула сложения вероятностей, независимые события}

Чтобы здесь хоть что-то решить, лучше полностью выучить теорию из всех прерыдущих глав.

\newpage
\subsection{Задание 31}

Тут ошибка в ответах!

n=36\\
A - на 1 кости четное\\
B - на 1 и 2 кости в сумме больше 3\\
Число исходов события В проще посчитать, если посчитать число исходов обратных В и вычесть это число из всех.
Всего исходов для $\overline{B}$:\\
11\\
12\\
21\\
Тогда,\\
$
n_B=36-3=33\\
n_A=3\cdot6=18\\
P(A)=\frac{18}{36}=\frac{1}{2}\\
P(B)=\frac{33}{36}=\frac{11}{12}\\
$\\
а)
$
A \cap B:\\
22,
23,
24,
25,
26,
\\
41,
42,
43,
44,
45,
46,
\\
61,
62,
63,
64,
65,
66.
\\
n_{A \cap B}=6+6+5=17\\
P(A \cap B) = \frac{17}{36}\\
$
б)
$
P(A \cup B) = P(A)+P(B)-P(A \cap B)=\frac{1}{2}+\frac{11}{12}-\frac{17}{36}=\frac{34}{36}=\frac{17}{18}
$\\
в)
$
P(A)=\frac{18}{36}=\frac{1}{2}
$\\
г)
$
P(\bar{A})=\frac{18}{36}=\frac{1}{2}
$\\
д)
$
n_{\overline{A \cap B}}=36-17=19
P(\overline{A \cap B})=\frac{19}{36}
$


\newpage
\subsection{Задание 32}

A - Анжи победит МЮ\\
B - Зенит победит Барселону\\
С -  наши победят\\
D - только одна наша команда победит\\ 
E - никто из наших не победит\\
F - выиграет только Зенит\\
$\\
P(A)=0.3\\
P(B)=0.4\\
P(C)=P(A \cap B)=0.3\cdot 0.4=0.12\\
P(D)=P(
(A \cap \overline{B})
 \cup
 (\overline{A} \cap {B})
)=P(A)\cdot P(\overline{B})+P(\overline{A})\cdot P( B)
=0.3\cdot 0.6+0.7\cdot 0.4=0.76\\
P(E)=P(\overline{A}\cap \overline{B})=P(\overline{A})\cdot P(\overline{B})=0.6\cdot 0.7=0.42\\
P(F)=P(\overline{A}\cap B)=0.28
$

\newpage
\subsection{Задание 33}

n=36\\
А - дубль\\
В - в сумме больше 9\\
$
P(A\cup B) - ?\\
n_A=6\\
n_B=6\\
A\cap B: 6-6, 5-5.\\
n_{A\cap B}=2\\
P(A)=\frac{6}{36}\\
P(B)=\frac{6}{36}\\
P(A\cap B)=\frac{2}{36}\\
P(A\cup B)=P(A)+P(B)-P(A\cap B)=
\frac{6}{36}+\frac{6}{36}-\frac{2}{36}=
\frac{10}{36}=\frac{5}{18}
$

\newpage
\subsection{Задание 34}

$
P(A\cap B)=P(A)\cdot P(B)\\
P(A)=0.4\\
P(B)=0.9\\
P(
(A \cap \overline B)
\cup 
(\overline  A\cap B)
\cup 
P(\overline B)\cdot P(\overline A)
)=P(A\cap \overline B)+
P(\overline A\cap B)+ 
P(\overline B)\cdot P(\overline A)=
P(A)\cdot P(\overline B) +  
P(\overline A)\cdot P( B)+
P(\overline A)\cdot P(\overline B)=
0.4\cdot 0.1+0.6\cdot 0.9+0.1\cdot 0.6=0.64 
$

\newpage
\subsection{Задание 35}

А - книга есть в первой библиотеке
В - книга есть во второй библиотеке
$
P(A)=0.7\\
P(B)=0.5\\
P(
(A\cap \overline B)\cup 
(\overline A\cap B)\cup 
(A\cap B)
)=
P(A)\cdot P(\overline B)+
P(\overline A)\cdot P(B)+
P(A)\cdot P(B)=
0.7\cdot 0.5+0.3\cdot 0.5 + 0.7\cdot 0.5 = 0.5\cdot(0.7+0.3+0.7)=
0.5\cdot 1.7=0.85
$

\newpage
\subsection{Задание 36}

$
P(A)=0.4\\
P(B)=0.7\\
P((A\cap \overline B)\cup (\overline A \cap B))=
P(A)\cdot P(\overline B)+
P(\overline A)\cdot P(B)=
0.4\cdot 0.3+0.6\cdot 0.7=0.54
$

\newpage
\subsection{Задание 37}

$
P(A)=0.6\\
P(B)=0.4\\
P((A\cap \overline B)\cup 
(\overline A \cap B)\cup 
(A\cap B))=
P(A)\cdot P(\overline B)+
P(\overline A)\cdot P(B)+
P(A)\cdot P(B)=
0.4\cdot 0.4+0.6\cdot 0.6 + 0.6 \cdot 0.4=0.76
$

\newpage
\subsection{Задание 38}

A - первый студент придет в срок\\
В - второй студент придет в срок\\
$
P(A)=0.8\\
P(B)=0.7\\
P((A\cap \overline B)\cup 
(\overline A \cap B))=
P(A)\cdot P(\overline B)+
P(\overline A)\cdot P(B)=
0.2\cdot 0.7+0.8\cdot 0.3=0.38
$

\newpage
\subsection{Задание 39}

А - увидеть на телевидении\\
В - прочитать в прессе\\
$
P(A)=0.7\\
P(B)=0.4\\
P(A\cap \overline B)=
0.7\cdot 0.6=0.42\\
P(A\cup B)=P(A)+P(B)-P(A\cap B)=
0.7+0.4-0.7\cdot 0.4=0.82
$

\newpage
\subsection{Задание 40}

А - отлично по первому предмету\\
В - отлично по второму предмету\\
$
P(A)=0.3\\
P(B)=0.5\\
P(A\cap B)=P(A)\cdot P(B)=0.3\cdot 0.5=0.15\\
P((A\cap \overline B)\cup 
(\overline A \cap B))=
P(A)\cdot P(\overline B)+
P(\overline A)\cdot P(B)=
0.3\cdot 0.5+0.7\cdot 0.5=0.5
$

\newpage
\subsection{Задание 41}

А - первый студент опоздает\\
В - второй студент опоздает\\
$
P(A)=0.2\\
P(B)=0.6\\
P((A\cap \overline B)\cup 
(\overline A \cap B)\cup 
(A\cap B))=
P(A)\cdot P(\overline B)+
P(\overline A)\cdot P(B)+
P(A)\cdot P(B)=
0.8\cdot 0.6+0.2\cdot 0.4 + 0.2 \cdot 0.6=0.68
$

\newpage
\section{\S Условная вероятность}

\subsection{Задание 42}

A - сумма очков бюольше 8\\
В - выпало четное число\\
$
n=36\\
n_{A\cap B}=3\\
n_B=9\\
P(B)=\frac{9}{36}\\
P(A\cap B)=\frac{3}{36}\\
p(A|B)=\frac{P(A\cap B)}{P(B)}=\frac{3}{36} : \frac{9}{36}=\frac{1}{3}
$

\newpage
\subsection{Задание 43}
A - на 1 любой выпало 6\\
В - на всех разные цифры\\
$
n=216\\
 n_B=6\cdot 5\cdot 4\\
P(B)=\frac{120}{216}\\
n_{A\cap B}=5\cdot 4 \cdot 3 = 60\\
P(A\cap B)=\frac{60}{216}\\
p(A|B)=\frac{P(A\cap B)}{P(B)}=\frac{60}{216} : \frac{120}{216}=\frac{1}{2}
$

\newpage
\subsection{Задание 44}
A - четное на 1\\
В - в сумме 8\\
$
n=36\\
 n_B=5\\
P(B)=\frac{5}{36}\\
n_{A}=6 \cdot 3 = 18\\
P(A)=\frac{18}{36}\\
n_{A\cap B}=3\\
P(A\cap B)=P(A)\cdot P(B)\\
\frac{3}{36} \neq \frac{18}{36} \cdot \frac{5}{36} \\
p(A|B)=\frac{P(A\cap B)}{P(B)}=\frac{3}{36} : \frac{36}{5}=\frac{6}{10}
$

\newpage
\subsection{Задание 45}
A - четное на 1\\
В - в сумме 8\\
$
n=8\\
 n_B=4\\
P(B)=\frac{4}{8}\\
n_{A}=4\\
P(A)=\frac{4}{8}\\
n_{A\cap B}=3\\
P(A\cap B)=P(A)\cdot P(B)\\
\frac{3}{8} \neq \frac{4}{8} \cdot \frac{8}{4} \\
p(A|B)=\frac{P(A\cap B)}{P(B)}=\frac{3}{36} : \frac{36}{5}=\frac{6}{10}\\
P(A\cup B)=P(A)+P(B)-P(A\cap B)=\frac{4}{8} + \frac{4}{8} - \frac{3}{8} = 0.685
$

\newpage
\subsection{Задание 46}

$
n=36\\
 n_B=18\\
P(B)=\frac{18}{36}\\
n_{A}=6\\
P(A)=\frac{6}{36}\\
n_{A\cap B}=4\\
P(A\cap B)=P(A)\cdot P(B)\\
\frac{4}{36} \neq \frac{18}{36} \cdot \frac{6}{36} \\
p(A|B)=\frac{P(A\cap B)}{P(B)}=\frac{4}{36} : \frac{36}{18}=\frac{2}{9}\\
P(A\cup B)=P(A)+P(B)-P(A\cap B)=\frac{6}{36} + \frac{18}{36} - \frac{4}{36} = \frac{20}{36}
$

\newpage
\subsection{Задание 47}

$
n=216\\
 n_B=10\\
P(B)=\frac{10}{216}\\
n_{A}=180\\
P(A)=\frac{180}{216}\\
n_{A\cap B}=8\\
P(A\cap B)=P(A)\cdot P(B)\\
\frac{8}{216} \neq \frac{10}{216} \cdot \frac{180}{216} \\
p(A|B)=\frac{P(A\cap B)}{P(B)}=\frac{8}{216} : \frac{216}{10}=\frac{8}{10}\\
$

\newpage
\subsection{Задание 48}

$
n=216\\
 n_B=108\\
P(B)=\frac{108}{216}\\
n_{A}=10\\
P(A)=\frac{10}{216}\\
n_{A\cap B}=7\\
p(A|B)=\frac{P(A\cap B)}{P(B)}=\frac{7}{216} : \frac{216}{108}=\frac{7}{108}\\
P(A\cup B)=P(A)+P(B)-P(A\cap B)=\frac{10}{216} + \frac{108}{216} - \frac{7}{216} = 
\frac{111}{216}=
\frac{37}{72}
$


\newpage
\subsection{Задание 49}

$
n=216\\
 n_B=4\\
P(B)=\frac{4}{216}\\
n_{A}=6\\
P(A)=\frac{6}{216}\\
n_{A\cap B}=1\\
P(A\cap B)=P(A)\cdot P(B)\\
\frac{1}{216} \neq \frac{6}{216} \cdot \frac{4}{216} \\
p(A|B)=\frac{P(A\cap B)}{P(B)}=\frac{1}{216} : \frac{216}{4}=\frac{1}{4}=0.25\\
$

\newpage
\subsection{Задание 50}

$
n=10\\
 n_B=6\\
P(B)=\frac{6}{10}\\
n_{A}=9\\
P(A)=\frac{9}{10}\\
n_{A\cap B}=6\\
P(A\cap B)=P(A)\cdot P(B)\\
\frac{6}{10} \neq \frac{9}{10} \cdot \frac{6}{10} \\
p(A|B)=\frac{P(A\cap B)}{P(B)}=\frac{6}{10} : \frac{10}{9}=\frac{2}{3}\\
$

\newpage
\subsection{Задание 51}
A - учебник у одного из друзей.\\
$
P(A)=\frac{8}{10}\\
P( \overline A)=\frac{2}{10}\\
A_1 - $учебник у Вани$\\
A_2 - $учебник у Пети$\\
P(A_1)=P(A_2)=\frac{4}{10}\\
$
В - у Вани учебника нет - учебник или у Пети или потерян\\
$
P(B)=P(A_2)+P(\overline A)=0.4-0.2=0.6\\
P($учебник у Пети, если у васи его нет$)=
\frac{P(A_2)}{P(B)}=\frac{0.4}{0.6}=\frac{2}{3}
$

\newpage
\section{\S Формулы полной вероятности и Байеса}

\subsection{Задание 52}
$
n=30\\
n_1=15\\
n_2=6\\
n_3=9\\
H_1-$первый сорт$\\
H_2-$второй сорт$\\
H_3-$третий сорт$\\
A-$червивое$\\
P(A|H_1)=0.2\\
P(A|H_2)=0.5\\
P(A|H_3)=0.1\\
P(H_1)=\frac{15}{30}\\
P(H_2)=\frac{6}{30}\\
P(H_3)=\frac{9}{30}\\
P(A)=P(A|H_1)\cdot P(H_1)+P(A|H_2)\cdot P(H_2)+P(A|H_3)\cdot P(H_3)\\
P(A)=0.2\cdot \frac{15}{30} + 0.5 \cdot \frac{6}{30} + 0.1 \cdot \frac{9}{30}=0.23
$

\newpage
\subsection{Задание 53}
$
n=150\\
n_1=75\\
n_2=60\\
n_3=15\\
H_1-$первая партия$\\
H_2-$вторая партия$\\
H_3-$третья партия$\\
A-$голосует "ЗА"$\\
P(A|H_1)=0.3\\
P(A|H_2)=0.4\\
P(A|H_3)=0.7\\
P(H_1)=\frac{75}{150}\\
P(H_2)=\frac{60}{150}\\
P(H_3)=\frac{15}{150}\\
P(A)=P(A|H_1)\cdot P(H_1)+P(A|H_2)\cdot P(H_2)+P(A|H_3)\cdot P(H_3)\\
P(A)=0.3\cdot \frac{75}{150} + 0.4 \cdot \frac{60}{150} + 0.7 \cdot \frac{15}{150}=0.38
$

\newpage
\subsection{Задание 54}
ЭТА ЗАДАЧА РЕШЕНА НЕВЕРНО\\
ЭТА ЗАДАЧА РЕШЕНА НЕВЕРНО\\
ЭТА ЗАДАЧА РЕШЕНА НЕВЕРНО\\
$
n=10\\
n_1=7\\
n_2=3\\
H_1-$в клетку$\\
H_2-$в линейку$\\
A-$забыл$\\
P(A|H_1)=\frac{1}{7}\\
P(A|H_2)=\frac{1}{3}\\
P(H_1)=\frac{7}{10}\\
P(H_2)=\frac{3}{10}\\
P(H_1|A)=\frac{P(A|H_1)\cdot P(H_1)}{P(A|H_1)\cdot P(H_1)+P(A|H_2)\cdot P(H_2)}\\
P(H_1|A)=\frac{\frac{1}{7} \cdot \frac{7}{10}}{\frac{1}{7} \cdot \frac{7}{10} + \frac{1}{3} \cdot \frac{3}{10}}=
$

\newpage
\subsection{Задание 55}
$
H_1-$благоприятная ситуация$\\
H_2-$неблагоприятная ситуация$\\
A-$продать$\\
P(A|H_1)=0.7\\
P(A|H_2)=0.2\\
P(H_1)=0.15\\
P(H_2)=0.85\\
P(A)=P(A|H_1)\cdot P(H_1)+P(A|H_2)\cdot P(H_2)\\
P(A)=0.7 \cdot 0.15 + 0.2 \cdot 0.85 = 0.275\\
P(H_2|A)=\frac{P(A|H_2)\cdot P(H_2)}{P(A|H_1)\cdot P(H_1)+P(A|H_2)\cdot P(H_2)}\\
P(H_2|A)=\frac{0.2 \cdot 0.85}{0.275}=\frac{34}{55}
$

\newpage
\subsection{Задание 56}

$
H_1-$опытный$\\
H_2-$неопытный$\\
A-$ошибка$\\
P(A|H_1)=0.02\\
P(A|H_2)=0.1\\
P(H_1)=0.9\\
P(H_2)=0.1\\
P(A)=P(A|H_1)\cdot P(H_1)+P(A|H_2)\cdot P(H_2)\\
P(A)=0.02 \cdot 0.9 + 0.1 \cdot 0.1 = 0.028\\
P(\overline A)=1-P(\overline A)\\
P(\overline A)=1-0.028=0.972\\
P(H_2|\overline A)=\frac{P(\overline A|H_2)\cdot P(H_2)}{P(\overline A)}\\
P(H_2|\overline A)=\frac{9}{10} \cdot \frac{1}{10} \cdot \frac{1000}{972}=\frac{5}{54}
$

\newpage
\subsection{Задание 57}

$
H_1-$первый округ$\\
H_2-$второй округ$\\
H_3-$третий округ$\\
A-$будет избран$\\
P(A|H_1)=0.4\\
P(A|H_2)=0.2\\
P(A|H_3)=0.8\\
P(H_1)=0.3\\
P(H_2)=0.2\\
P(H_2)=0.5\\
P(A)=P(A|H_1)\cdot P(H_1)+P(A|H_2)\cdot P(H_2)+P(A|H_3)\cdot P(H_3)\\
P(A)=0.4 \cdot 0.3 + 0.2 \cdot 0.2 + 0.8 \cdot 0.5 =0.56
$

\newpage
\subsection{Задание 58}
$
H_1-$отлично по математике на первом курсе$\\
H_2-$неотлично по математике на первом курсе$\\
A-$отлично по математике на втором курсе$\\
P(A|H_1)=0.8\\
P(A|H_2)=0.15\\
P(H_1)=0.1\\
P(H_2)=0.9\\
P(A)=P(A|H_1)\cdot P(H_1)+P(A|H_2)\cdot P(H_2)\\
P(A)=0.8 \cdot 0.1 + 0.15 \cdot 0.9 = 0.215
$

\newpage
\subsection{Задание 59}
$
H_1-$отлично по математике на первом курсе$\\
H_2-$неотлично по математике на первом курсе$\\
A-$отлично по математике на втором курсе$\\
P(A|H_1)=0.7\\
P(A|H_2)=0.25\\
P(H_1)=0.2\\
P(H_2)=0.8\\
P(A)=P(A|H_1)\cdot P(H_1)+P(A|H_2)\cdot P(H_2)\\
P(A)=0.7 \cdot 0.2 + 0.25 \cdot 0.8 = 0.34\\
$
Значит, 34 процента

\newpage
\subsection{Задание 60}
$
H_1-$горожанин$\\
H_2-$сельчанин$\\
A-$голос за Единую Россию$\\
P(A|H_1)=0.4\\
P(A|H_2)=0.6\\
P(H_1)=0.75\\
P(H_2)=0.25\\
P(A)=P(A|H_1)\cdot P(H_1)+P(A|H_2)\cdot P(H_2)\\
P(A)=0.4 \cdot 0.75 + 0.6 \cdot 0.25 = 0.45\\
P(\overline A)=1-P(A) = 0.55
$

\newpage
\subsection{Задание 61}
$
H_1-$юноши$\\
H_2-$девушки$\\
A-$отлично на экзамене по тер. веру$\\
P(A|H_1)=0.2\\
P(A|H_2)=0.3\\
P(H_1)=0.25\\
P(H_2)=0.75\\
P(A)=P(A|H_1)\cdot P(H_1)+P(A|H_2)\cdot P(H_2)\\
P(A)=0.2 \cdot 0.25 + 0.3 \cdot 0.75 =0.275
$

\newpage
\subsection{Задание 62}
$
P($брак проверен и признан годным(один поставщик)$)=0.05\\
P($годное проверено и признано браком(другой поставщик)$)=0.01\\
P($произвести брак$)=0.03\\
P($произвести годное$)=0.97\\
P($брак проверен и признан браком$)=1-0.05=0.95\\
P($годный проконтроллирован и признан браком(считай изделия от первого поставщика оказалось бюраком)$)=0.97\cdot 0.01=0.0097\\
P($брак проконтроллирован и признан браком(считай изделия от второго поставщика оказалось бюраком)$)=0.03\cdot 0.95 = 0.0285\\
P($Проконтроллирован брак в любом случае(полная вероятность)$)=0.0097+0.0285=0.0382\\
P($контроль забраковал, а но на самом деле годное(считаем брак,  какой вероятностью это второй поставщик)$)=\frac{0.0097}{0.0382}=\frac{97}{382}
$


\newpage
\subsection{Задание 63}
$
H_1-$1 производитель$\\
H_2-$2 производитель$\\
H_3-$3 производитель$\\
P(H_1)=0.2\\
P(H_2)=0.45\\
P(H_3)=0.35\\
A-$брак$\\
P(A|H_1)=0.08\\
P(A|H_2)=0.02\\
P(A|H_3)=0.05\\
P(A)=P(A|H_1)\cdot P(H_1)+P(A|H_2)\cdot P(H_2)+P(A|H_3)\cdot P(H_3)\\
P(A)=0.08 \cdot 0.2 + 0.02 \cdot 0.45 + 0.05 \cdot 0.35=0.0425\\
P(H_2|A)=\frac{P(A|H_2)\cdot P(H_2)}{P(A)}\\
P(H_2|A)=\frac{0.02 \cdot 0.45}{0.0425}=\frac{90}{485}=\frac{18}{85}
$

\newpage
\subsection{Задание 64}
$
H_1-$отличники$\\
H_2-$хорошисты$\\
H_3-$середнячки$\\
P(H_1)=15/60\\
P(H_2)=27/60\\
P(H_3)=18/60\\
A-$оценть положительно качество преподавания$\\
P(A|H_1)=0.95\\
P(A|H_2)=0.9\\
P(A|H_3)=0.5\\
P(\overline A|H_1)=0.05\\
P(\overline A|H_2)=0.1\\
P(\overline A|H_3)=0.5\\
P(\overline A)=P(\overline A|H_1)\cdot P(H_1)+P(\overline A|H_2)\cdot P(H_2)+P(\overline A|H_3)\cdot P(H_3)\\
P(\overline A)=0.05\cdot \frac{15}{60}+0.1 \cdot \frac{27}{60} + 0.5 \cdot \frac{18}{60}=\frac{1245}{6000}\\
P(H_3|\overline A)=\frac{P(\overline  A|H_3)\cdot P(H_3)}{P(\overline A)}\\
P(H_3|\overline A)=\frac{0.5 \cdot \frac{18}{60}}{\frac{1245}{6000}}=60/83\\
$

\newpage
\subsection{Задание 65}

Боже это слишком сложно

\newpage
\subsection{Задание 66}
$
H_1-$1 поставщик$\\
H_2-$2 поставщик$\\
H_3-$3 поставщик$\\
P(H_1)=0.6\\
P(H_2)=0.3\\
P(H_3)=0.1\\
A-$Брак$\\
P(A|H_1)=0.05\\
P(A|H_2)=0.1\\
P(A|H_3)=0.01\\
P(\overline A|H_1)=0.95\\
P(\overline A|H_2)=0.9\\
P(\overline A|H_3)=0.99\\
P(\overline A)=P(\overline A|H_1)\cdot P(H_1)+P(\overline A|H_2)\cdot P(H_2)+P(\overline A|H_3)\cdot P(H_3)\\
P(\overline A)=0.95\cdot 0.6+0.9 \cdot 0.3 + 0.99 \cdot 0.1=\frac{939}{1000}\\
P(H_2|\overline A)=\frac{P(\overline  A|H_2)\cdot P(H_2)}{P(\overline A)}\\
P(H_2|\overline A)=\frac{0.9 \cdot 0.3}{\frac{939}{1000}}=90/313\\
$

Остальные задачи в этой теме показались мне ахинеей, может потом я пойму, как их решать. Но пока что...следующая тема.

\newpage
\section{\S Испытание Бернулли. Биномиальное распределение}

\subsection{Задание 69}
p=0.3\\
q=0.7\\
n=5\\
k=3\\
$
P(S_k=3)=C_{5}^{3}\cdot 0.3^3\cdot 0.7^2 = 0.2323 
$
\newpage
\subsection{Задание 70}
p=0.5\\
q=0.5\\
n=4\\
k=0,1\\
$
C_{n}^{0} = 1\\
P(S_k<2)=C_{4}^{0}\cdot p^0\cdot q^4 + C_{4}^{1}\cdot p^1\cdot q^3 =  \frac{5}{16} 
$
\newpage
\subsection{Задание 71} 
p=0.95\\
q=0.05\\
n=4\\
k=1-4\\
Обратить внимание на НЕ в условии\\
Это меняет p и q  местами\\
p=0.05\\
q=0.95\\
$
P(S_k>0)=1-P(S_k=0)=1-(C_{4}^{0}\cdot 0.05^0 \cdot 0.95^4) = 1-  0.95^4 
$
\newpage
\subsection{Задание 72}
p=1/4\\
q=3/4\\
n=5\\
k=2-5\\
$
P(S_k>1)=1-P(S_k<2)=1-(C_{5}^{1}\cdot (1/4)^1\cdot (3/4)^4 + C_{5}^{0}\cdot (1/4)^0\cdot (3/4)^5) =  \frac{47}{128} 
$
\newpage
\subsection{Задание 73}
p=1/6\\
q=5/6\\
n=4\\
k=0-2\\
$
P(S_k<3)=C_{4}^{0}\cdot p^0\cdot q^4 + C_{4}^{1}\cdot p^1\cdot q^3  + C_{4}^{2}\cdot p^2\cdot q^2=  \frac{425}{432} 
$
\newpage
\subsection{Задание 74}
p=0.8\\
q=0.2\\
n=4\\
k=2-4\\
$
P(S_k>1)=1-P(S_k<2)=1-(C_{4}^{0}\cdot p^0\cdot q^4 + C_{4}^{1}\cdot p^1\cdot q^3) =  \frac{608}{625} 
$
\newpage
\subsection{Задание 75}
p=0.1\\
q=0.9\\
n=5\\
k=2-5\\
$
P(S_k>1)=1-P(S_k<2)=1-(C_{5}^{1}\cdot p^1\cdot q^4 + q^5) =  0.08146
$
\newpage
\subsection{Задание 76}
ответ на эту задачу в 72 задаче - то, что вычитается из 1.\\
$
P(S_k<2)=C_{5}^{1}\cdot (1/4)^1\cdot (3/4)^4 + C_{5}^{0}\cdot (1/4)^0\cdot (3/4)^5=\frac{81}{128}
$
\newpage
\subsection{Задание 77}
p=1/6\\
q=5/6\\
n=4\\
k=2-4\\
$
P(S_k>1)=1-P(S_k<2)=1-(C_{4}^{0}\cdot p^0\cdot q^4 + C_{4}^{1}\cdot p^1\cdot q^3) =  \frac{19}{144} 
$
\newpage
\subsection{Задание 78}
p=1/4\\
q=3/4\\
n=5\\
1)\\
k=3-5\\
$
P(S_k>2)=C_{5}^{3}\cdot p^3\cdot q^2 + C_{4}^{5}\cdot p^4\cdot q^1+ C_{5}^{5}\cdot p^5\cdot q^0 =  \frac{53}{512} 
$\\
2)\\
k=5\\
$
P(S_k=5)=C_{5}^{5}\cdot p^5\cdot q^0=  \frac{1}{1024} 
$
\newpage
\subsection{Задание 79}
Всегда путлася, хотя бы 1 неуспешный случай  - это все кроме только успехов.\\
$
1-0.8^5=1-\frac{4}{5}^5
$
\newpage
\subsection{Задание 80}
p=0.01\\
q=0.99\\
n=8\\
a)\\
k=2\\
$
P(S_k=2)=C_{8}^{2}\cdot p^2\cdot q^6 =  28 \cdot 0.01^2\cdot 0.99^6
$\\
б)\\
хотя бы 1 выиграть - все случаи, кроме только проигрышей\\
$
1-0.99^8
$
\newpage
\subsection{Задание 81}
p=0.8\\
q=0.2\\
n=5\\
а)\\
k=5\\
$
P(S_k=5)=C_{5}^{5}\cdot p^5\cdot q^0=0.8^5=0.33
$\\
б)\\
k=2\\
$
P(S_k=2)=C_{5}^{2}\cdot p^2\cdot q^3=\frac{32}{625}=0.05
$
\newpage
\subsection{Задание 82}
p=0.95\\
q=0.05\\
n=4\\
а)\\
1-(вероятность случая, когда все удов.)\\
$
P(S_k=3)=C_{4}^{0}\cdot p^0\cdot q^4 + C_{4}^{1}\cdot p^1\cdot q^3 =  \frac{5}{16} 
$\\
б)\\
$
C_{4}^{1}\cdot p^1\cdot q^3 =  0.000475 
$

\newpage
\section{\S Дискретная случаная величина и ее числовые характеристики: математические ожидание и дисперсия}

\subsection{Задание 83}
Тут главное помнить, что в сумме все значения p из таблицы должны быть равны 1.\\
Отсюда \\
$
p(2)=1 - (p(-1) + p(0)+ p(4)) =
  1-(0.1+0.1+0.3)=0.5\\
E(X) = 0.1 * (-1) + 0.1 * 0 + 0.5*2  +  0.3*4 = 2.1\\
E(X^{2}+1) = 0.1 * 2 + 0.1 * 1 + 0.5*5  + 17*0.3 = 7.9\\
$
\newpage
\subsection{Задание 84}
Тут главное помнить, что в сумме все значения p из таблицы должны быть равны 1.\\
Отсюда 
$
p(0)=1 - (p(-1) + p(2)+ p(3)) =  0.4\\
E(X) = 0.2 * (-1) + 0.1 * 2 + 0.3 * 3 = 0.9\\
E(3X-5) = 0.2 * (-8) + 0.4 * (-5)+ 0.1*1  + 4*0.3 = -2.3\\
$
\newpage
\subsection{Задание 85}
\begin{tabular}{|c|c|c|c|c|c|}
\hline
Номинал: & \$1 & \$5 & \$20 & \$50 & \$100 \\
\hline
Кол-во: & 10 & 5 & 3 & 1 & 1 \\
\hline
\end{tabular}
\\
\\
\begin{tabular}{|c|c|c|c|c|c|}
\hline
Сумма выигрыша: & \$ -19 & \$ -15 & \$ 0 & \$30 & \$80 \\
\hline
Кол-во: & 0.5 & 0.25 & 0.15 & 0.05 & 0.05 \\
\hline
\end{tabular}
\\
\\
$
E(X)=-19*0.5-15*0.25+30*0.05+80*0.05=-7.75\\
E(X^2)=-19^2*0.5-15^2*0.25+30^2*0.05+80^2*0.05=601,75\\
E^2(X)=60.0625\\
D(X)=E(X^2)-E^2(X)\\
$
Тут падла от автора, ответ умножили на 16 и оставили в виде дроби
$
D(X)=601.75-60.0625=541.6875=\frac{541.6875\cdot 16}{16}=\frac{8667}{16}\\
$
\newpage
\subsection{Задание 86}
\begin{tabular}{|c|c|c|c|c|c|}
\hline
X: & 1 & 2 & 3 & ... & n \\
\hline
P: & 1/2 & 1/4 & 1/8 & ... &         $1/2^n        $ \\
\hline
\end{tabular}
\\
\\
Тут пришлось подсмотреть ответы. В условии не сказано ни одной вводной цифры, но цифры в ответе есть. Значит, до всего нужно догадывться. Первое - число экспериментов от 1 до бесконечности(до n, это теперь понятно, но без ответов не смог понять).\\
Обычно все неизвестные необходимые входные параметры помечаются буквами и в ответе получается формула.\\
Тут вероятность выпадения решки = 1/2. Сторон всего 2, нас интересует только одна.\\
Не понятно, почему при увеличении n, $p = 1/2^n$, то есть вероятности на предыдущих n умножаются на текущую.\\
Версия первая - это независимые события поэтому их вероятности перемножаются.\\
Версия вторая  - это эксперимент бернули, успешный всегда последний бросок, остальные неуспешные. Это как писать тест, в котором количество вопросов увеличивается на 1. Вы всегда 1 раз отвечаете правильно и n-1 раз неправильною\\
Поэтому 1/2 (вероятность успеха) умножается  на вероятность 1/2 (вероятность неудачи) в степени n-1 (по формуле Бернули все дела). Отсюда $1/2^n$.\\
Дальше непонятно почему EX = 2.\\
У меня:\\
$
EX = \sum_{i=1}^n n\cdot \frac{1}{2^n}
$
\newpage
\subsection{Задание 87}
Номинал : Кол-во\\
1000=1\\
500=1\\
100=5\\
3=993\\
\\
\begin{tabular}{|c|c|c|c|c|}
\hline
Выигрыш: & -3 & 97 & 497 & 997  \\
\hline
P: & 993/1000 & 5/1000 & 1/1000 & 1/1000 \\
\hline
\end{tabular}
\\
\\
$
p(>200)= p(500)+p(1000)=0.001+0.001=0.002\\
EX = \frac{993*(-3)}{1000}+\frac{5*97}{1000}+\frac{497}{1000}+\frac{997}{1000}=-1
$
\newpage
\subsection{Задание 88}

-10=481\\
990=1\\
240=8\\
90=10\\
\\
\begin{tabular}{|c|c|c|c|c|}
\hline
X: & -10 & 90 & 240 & 990  \\
\hline
P/2: & 481/500 & 10/500 & 8/500 & 1/500 \\
\hline
P: & 962/1000 &$ \frac{20}{1000}=0.02$ & 16/1000 & 2/1000 \\
\hline
\end{tabular}
\\
\\
$
p(>100)= p(240)+p(990)=0.016+0.002=0.018\\
EX = \frac{962*(-10)}{1000}+\frac{2*90}{100}+\frac{16*240}{1000}+\frac{2*990}{1000}=-2
$

\newpage
\subsection{Задание 89}

72 задача как подсказка

\newpage
\section{\S Распределение Пуассона}
$
P(x=k)=\frac{\lambda^k}{k!}e^{-\lambda}
$
\subsection{Задание 90}
$
\lambda=2\\
P(X<3)=P(0)+P(1)+P(2)=
\frac{2^0}{0!}e^{-2}+\frac{2^1}{1!}e^{-2}+\frac{2^2}{2!}e^{-2}=
e^{-2}+2e^{-2}+2e^{-2}=5e^{-2}
$
\newpage
\subsection{Задание 91}
$
\lambda=1\\
P(X>1)=1-P(X<2)=1-P(0)-P(1)=
1-\frac{1^0}{0!}e^{-1}-\frac{1^1}{1!}e^{-1}=
1-2e^{-1}
$
\newpage
\subsection{Задание 92}
В каждой задаче нужно все приводить к одной единице измерений.\\
24 в час, в часе 60 минут, в 60 минутах 12 кусочков, в 1 кусочке(5 минутах) 2 звонка\\
$
\lambda=2\\
P(X>3)=1-P(0)-P(1)-P(2)-P(3)=
$\\
Из 90 задачи знаем, что $P(0)+P(1)+P(2)=5e^{-2}$\\
$
1-5e^{-2}-\frac{2^3}{3!}e^{-2}=1-\frac{19}{3}e^{-2}
$
\newpage
\subsection{Задание 93}
За 4 месяца будет 2 аварии.\\
$
\lambda=2\\
P(X>2)=1-P(0)-P(1)-P(2)=
$\\
Из 90 задачи знаем, что $P(0)+P(1)+P(2)=5e^{-2}$\\
$
1-5e^{-2}
$
\newpage
\subsection{Задание 94}
В минуту будет 2 звонка.\\
$
\lambda=2\\
P(2,3)=P(2)+P(3)=2e^{-2}+\frac{4}{3}e^{-2}=\frac{10}{3}e^{-2}
$
\newpage
\subsection{Задание 95}
$
\lambda=4\\
P(X<3)=P(0)+P(1)+P(2)=e^{-4}+4e^{-4}+8e^{-4}=13e^{-4}
$
\newpage
\subsection{Задание 96}
$
\lambda=4\\
P(X>2)=1-(P(0)+P(1)+P(2))=1-13e^{-4}
$
\newpage
\subsection{Задание 97}
12 в час => 12/12 => 1 в 5 минут\\
Тут подсмотрим 91 задачу:$P(0)+P(1)=2e^{-1}$\\
$
\lambda=1\\
P(X>2)=1-(P(0)+P(1)+P(2))=1-2e^{-1}-\frac{1}{2}e^{-1}=1-\frac{5}{2}e^{-1}
$
\newpage
\section{\S Совместное распределение двух дискретных величин. Ковариация и корелляция двух случайных величин}
\subsection{Задание 98}
\begin{tabular}{|c|c|c|}
\hline
X: & 0 & 1  \\
\hline
P: & 0.1 & 0.9 \\
\hline
\end{tabular}
\\
\\
\begin{tabular}{|c|c|c|c|}
\hline
Y: & -1 & 0 & 1  \\
\hline
P: & 0.2 & 0.3 & 0.5 \\
\hline
\end{tabular}
\\
\\
$X/Y = X\cdot Y$\\
\\
\begin{tabular}{|c|c|c|c|}
\hline
X/Y: & -1 & 0 & 1  \\
\hline
0: & 0.02 & 0.03 & 0.05 \\
\hline
1: & 0.18 & 0.27 & 0.45 \\
\hline
\end{tabular}
\\ 
\newpage
\subsection{Задание 99}
P(X=0)=1/2\\
P(X=1)=1/2\\
P(Y=0)=5/6\\
P(Y=1)=1/6\\
\\
$X/Y=X\cdot Y:$\\
P(0,1)=1/12\\
P(1,0)=5/12\\
P(1,1)=1/12\\
ну и так далее\\
\\
\begin{tabular}{|c|c|c|}
\hline
X/Y: & 0 & 1  \\
\hline
0: & 5/12 & 1/12 \\
\hline
1: & 5/12 & 1/12 \\
\hline
\end{tabular}
\\
\newpage
\subsection{Задание 100}
X - монеты, Y - пятерки.\\
\\
\begin{tabular}{|c|c|c|c|}
\hline
X: & 0 & 1& 2  \\
\hline
P: & 1/4 & 2/4 & 1/4 \\
\hline
\end{tabular}
\\
\\
\begin{tabular}{|c|c|c|c|}
\hline
Y: & 0 & 1& 2  \\
\hline
P: & 25/36 & 5/36 & 1/36 \\
\hline
\end{tabular}
\\
\\
$X/Y=X\cdot Y:$\\
P(0,2)=1/4 * 1/36 = 1/144\\
P(1,0)=2/4 * 25/36 = 50/144\\
ну и так далее\\
\\
\begin{tabular}{|c|c|c|c|}
\hline
X/Y: & 0 & 1 & 2 \\
\hline
0: & 25/144 & 5/144 & 4/155 \\
\hline
1: & 50/144 & 10/144 & 2/144 \\
\hline
2: & 25/144 & 5/144 & 1/144 \\
\hline
\end{tabular}
\\
\newpage
\subsection{Задание 101}
Почему-то из всего не вычитается страховой взнос, хотя смысл имеет.\\
Для 1 договора(Х):\\
\\
\begin{tabular}{|c|c|c|c|}
\hline
X(тыс.руб.): & 0 & 50& 100  \\
\hline
P: & 0.94 & 0.05 & 0.01 \\
\hline
\end{tabular}
\\
\\
Для 2 договора(Y):\\
\\
\begin{tabular}{|c|c|c|c|}
\hline
Y(тыс.руб.): & 0 & 50& 100  \\
\hline
P: & 0.94 & 0.05 & 0.01 \\
\hline
\end{tabular}
\\
\\
\begin{tabular}{|c|c|c|c|}
\hline
X/Y:  & 0 & 50& 100  \\
\hline
0: &  0.94*0.94 & 0.05*0.94 & 0.01*0.94 \\
\hline
1: & 0.94*0.05 & 0.05*0.05 & 0.01*0.05 \\
\hline
2: &  0.94*0.01 & 0.05*0.01 & 0.01*0.01 \\
\hline
\end{tabular}
\\
\\
В ответах ошибка для 50/50\\
\newpage
\subsection{Задание 102}
x(1)*y(1)=0.5*0.6=0.3\\
х*у =0 включает 3 исхода, проще невычислять их, а сделать 1-(х*у=1)(включает только x(1)*y(1))\\
\\
\begin{tabular}{|c|c|c|}
\hline
X*Y: & 0 & 1  \\
\hline
P: & 1-0.3 & 0.3  \\
\hline
\end{tabular}
\\
\\
$E(x*y)=0*0.7+1*0.3=0.3$\\
\\
\begin{tabular}{|c|c|c|}
\hline
X: & 0 & 1  \\
\hline
P: & 0.5 & 0.5  \\
\hline
\end{tabular}
\\
\\
\begin{tabular}{|c|c|c|}
\hline
Y: & 0 & 1  \\
\hline
P: & 0.4 & 0.6  \\
\hline
\end{tabular}
\\
\\
$
E(x)=0.5\\
E(y)=0.6\\
E(x)*E(y)=0.3\\
E(x)*E(y) = E(x*y)\\
$ \\
Условие независимости выполняется, значит независимы.\\
\newpage
\subsection{Задание 103}
\begin{tabular}{|c|c|c|}
\hline
X: & -1 & 1  \\
\hline
P: & 0.4 & 0.6  \\
\hline
\end{tabular}
\\
\\
\begin{tabular}{|c|c|c|c|}
\hline
Y: & -1 & 0 & 2  \\
\hline
P: & 0.2 & 0.3 & 0.5  \\
\hline
\end{tabular}
\\
\\
Вычисляем величину X*Y:\\
P((x=-1)*(y=-1))=P(XY=1)=0.2*0.4=0.08\\
P((x=-1)*(y=0))=P(XY=0)=0.3*0.4=0.12\\
P((x=-1)*(y=2))=P(XY=-2)=0.4*0.5=0.2\\
P((x=1)*(y=-1))=P(XY=-1)=0.6*0.2=0.12\\
P((x=1)*(y=0))=P(XY=0)=0.6*0.3=0.18\\
P((x=1)*(y=2))=P(XY=2)=0.6*0.5=0.3\\
\\
P(XY=-2)=P((x=-1)*(y=2))\\
P(XY=-1)=P((x=1)*(y=-1))\\
P(XY=0)=P((x=-1)*(y=0))+P((x=1)*(y=0))\\
P(XY=1)=P((x=-1)*(y=-1))\\
P(XY=2)=P((x=1)*(y=2))\\
\\
\begin{tabular}{|c|c|c|c|c|c|}
\hline
XY: & -2 & -1 & 0 & 1 & 2  \\
\hline
P: & 0.2 & 0.12 & 0.3 & 0.08 & 0.3  \\
\hline
\end{tabular}
\\
\\
E(XY)=-2*0.2-1*0.12+0.08+0.3*2=-0.4-0.12+0.08+0.6=0.2-0.04=0.16\\
E(X)=0.2\\
E(Y)=0.8\\
E(XY)=E(X)E(Y) => независимы\\

\newpage
\subsection{Задание 104}
\begin{tabular}{|c|c|c|}
\hline
X: & 0 & 2  \\
\hline
P: & 0.2 & 0.8  \\
\hline
\end{tabular}
\\
\\
\begin{tabular}{|c|c|c|c|}
\hline
Y: & 1 & 2 & 3  \\
\hline
P: & 0.5 & 0.1 & 0.4  \\
\hline
\end{tabular}
\\
\\
\begin{tabular}{|c|c|c|c|}
\hline
X/Y: & 1 & 2 & 3  \\
\hline
0: & 0.1 & 0 & 0.1  \\
\hline
2: & 0.4 & 0.1 & 0.3  \\
\hline
\end{tabular}
\\
Возможно дело вот в чем - если дано совместное распределение, то необходимо брать его как исходные данные для нахождения величин типа XY или X+Y. А если не задано, то использовать ряды распределения каждой величины в отдельности.\\
\\
Для X+Y = 3 есть 2 исхода: X=0/Y=3 и X=2/Y=1, то есть 0.1 + 0.4, отсюда  и 0.5. В остальных случаях по одному исходу.\\
\\
\begin{tabular}{|c|c|c|c|c|c|}
\hline
X+Y: & 1 & 2 & 3 & 4 & 5 \\
\hline
P: & 0.1 & 0 & 0.5 & 0.1 & 0.3 \\
\hline
\end{tabular}
\\
Теперь для XY:\\
X | Y\\
0 * 1 = 0\\
0 * 2 = 0\\
0 * 3 = 0\\
2 * 1 = 2\\
2 * 2 = 4\\
2 * 3 = 6\\
\\
\begin{tabular}{|c|c|c|c|c|}
\hline
XY: & 0 & 2 & 4 & 6 \\
\hline
P: & 0.2 & 0.4 & 0.1 & 0.3 \\
\hline
\end{tabular}
\\
E(XY)=0.8+0.4+1.8=3\\
E(X)=1.6\\
E(Y)=1.7\\
E(XY) не равно E(X)*E(Y) - величины зависимы.
\newpage
\subsection{Задание 105}
Возможно дело вот в чем - если дано совместное распределение, то необходимо брать его как исходные данные для нахождения величин типа XY или X+Y. А если не задано, то использовать ряды распределения каждой величины в отдельности.\\
\\
P(x=1,y=2)=0\\
\begin{tabular}{|c|c|c|}
\hline
X: & -2 & 1  \\
\hline
P: & 0.8 & 0.2  \\
\hline
\end{tabular}
\\
\\
\begin{tabular}{|c|c|c|c|}
\hline
Y: & -1 & 0 & 2  \\
\hline
P: & 0.3 & 0.3 & 0.4  \\
\hline
\end{tabular}
\\
E(X)=-1.6+0.2=-1.4\\
E(Y)=-0.3+0.8=0.5\\
\\
Для нахождения ряда распределения XY надо использовать X/Y, ни в коем случае не брать отдельно распределения X и Y. Хотя в задачах, когда изначально известны распределения X и Y отдельно и больше ничего, берут именно их за основу.\\
\\
Найдем все возможные значения XY:\\
-2 * -1=2\\
-2 * 0=0\\
-2 * 2=-4\\
1 * -1=-1\\
1 * 0=0\\
1 * 2=2\\
Получаем два исхода для XY=2 и XY = 0.\\
На примере XY=2 , это будет 0.1(X/Y для X=-2 и Y=-1) плюс 0(X/Y для X=1 и Y=2)\\
\\
\begin{tabular}{|c|c|c|c|c|}
\hline
XY: & -4 & -1 & 2 & 0\\
\hline
P: & 0.4 & 0.2 & 0.1 & 0.3  \\
\hline
\end{tabular}
\\ 
\\
E(XY)=-4*0.4-0.2+0.2=-1.6\\
E(XY) не равно E(X) + E(Y), значит эти величины НЕ являются независимыми.
\newpage
\subsection{Задание 106}
\begin{tabular}{|c|c|c|c|}
\hline
X: & 0 & 1 & 2 \\
\hline
P: & 1/4 & 3/8 & 3/8  \\
\hline
\end{tabular}
\\ 
\\
EX=3/8 + 6/8= 9/8\\
\\
\begin{tabular}{|c|c|c|c|}
\hline
X*X-3: & -3 & -2 & 1 \\
\hline
P: & 1/4 & 3/8 & 3/8  \\
\hline
\end{tabular}
\\ 
\\
E(X*X-3)= -3/4-6/8+3/8=-9/8\\
\\
\begin{tabular}{|c|c|c|c|}
\hline
X*X: & 0 & 1 & 4 \\
\hline
P: & 1/4 & 3/8 & 3/8  \\
\hline
\end{tabular}
\\ 
\\
E(X*X) = 3/8 + 12/8 = 15/8\\
DX = 15/8-81/64=39/64\\
\\
\\
\begin{tabular}{|c|c|c|c|c|}
\hline
X*Y: & 0 & -4 & -2& -1 \\
\hline
P: & -7/16 & 5/16 & 1/16 & 3/16  \\
\hline
\end{tabular}
\\ 
\\
\\
\begin{tabular}{|c|c|c|c|}
\hline
Y: & -2 & -1 & 0 \\
\hline
P: & 1/2 & 1/4 & 1/4  \\
\hline
\end{tabular}
\\ 
\\
EY=-5/4\\
E(XY)=-20/16-2/16/-3/16=-25/16\\
Cov(XY)=-25/16-(9/8*(-5/4))=-5/32\\
\newpage
\subsection{Задание 107}
\begin{tabular}{|c|c|c|c|}
\hline
X/Y: & -1 & 0 & 1 \\
\hline
-0.5: & 3/25 & 2/25 & 0  \\
\hline
0: & 7/25 & 3/25 & 3/25  \\
\hline
0.5: & 0 & 5/25 & 2/25  \\
\hline
\end{tabular}
\\
\\
\begin{tabular}{|c|c|c|c|}
\hline
Y: & -1 & 0 & 1 \\
\hline
P: & 10/25 & 10/25 & 5/25  \\
\hline
\end{tabular}
\\
\\
\begin{tabular}{|c|c|c|c|}
\hline
X: & -0.5 & 0 & 0.5 \\
\hline
P: & 5/25 & 13/25 & 7/25  \\
\hline
\end{tabular}
\\    
EY=-10/25+5/25=-0.2\\
\\
\\
\begin{tabular}{|c|c|c|c|}
\hline
Y*Y: & 1 & 0 & 1 \\
\hline
P: & 10/25 & 10/25 & 5/25  \\
\hline
\end{tabular}
\\
\\
E(Y*Y)=15/25=60/100\\
DY=0.6-0.04=0.56\\
\\
\begin{tabular}{|c|c|c|c|}
\hline
Y*Y*Y-4: & -5 & -4 & -3 \\
\hline
P: & 10/25 & 10/25 & 5/25  \\
\hline
\end{tabular}
\\
\\
E(Y*Y*Y-4)=-21/5=-4.2\\
Y * X:\\
-1 * (-0.5)=0.5\\
0 * (-0.5)=0\\
1 * (-0.5)=-0.5\\
-1 * 0=0\\
0 * 0=0\\
1 * 0=0\\
-1 * 0.5=-0.5\\
0 * 0.5=0\\
1 * 0.5=0.5\\
\\
\begin{tabular}{|c|c|c|c|}
\hline
X*Y & -0.5 & 0 & 0.5 \\
\hline
P: & 0 & 20/25 & 5/25  \\
\hline
\end{tabular}
\\
\\
E(X)=0.04\\
E(XY)=0.5*5/25 = 1/10\\
DY=0.56\\
Cov(X,Y)=1/10+0.04*0.2=0.108\\
\\
\begin{tabular}{|c|c|c|c|}
\hline
X*X & 0.25 & 0 & 0.25 \\
\hline
P: & 5/25 & 13/25 & 7/25  \\
\hline
\end{tabular}
\\
E(X*X)=5/100+7/100=12/100\\
DX=12/100-16/10000=0.1184\\
Cor(X,Y)=$\frac{0.108}{\sqrt{0.56} \cdot \sqrt{0.1184}}=$0.108/0.257=0.42023\\
\newpage
\subsection{Задание 108}
Тут в лоб решать мутрно, решил воспользоваться некоторыми свойствами дисперсии и мат.ожидания.\\
Лучше их тупо знать наизусть, чтобы понять решение.\\
$
D(3X-2Y)=
D(3X)+D(-2Y)+2COV(3X,-2Y)=
9D(X)+4D(Y)+E(3X*(-2Y))-E(3X)*E(-2Y)=
9D(X)+4D(Y)+E(3X*(-2Y))+6E(X)*E(Y)\\
$\\
EY=-0.2\\
DY=0.56\\
EX=0.04\\
E(XY)=1/10\\
DX=0.1184\\
\\
\\
\begin{tabular}{|c|c|c|c|}
\hline
3X & -1.5 & 0 & 1.5 \\
\hline
P: & 5/25 & 13/25 & 7/25  \\
\hline
\end{tabular}
\\
\\
\begin{tabular}{|c|c|c|c|}
\hline
-2Y: & 2 & 0 & -2 \\
\hline
P: & 10/25 & 10/25 & 5/25  \\
\hline
\end{tabular}
\\
3X*(-2Y):\\
-1.5 * 2 = -3\\
0 * 2 = 0\\
1.5 * 2 = 3\\
-1.5 * 0 = 0\\
0 * 0 = 0\\
1.5 * 0 = 0\\
-1.5 * (-2) = 3\\
0 * (-2) = 0\\
1.5 * (-2) = -3\\
\\
$
P(3X*(-2Y)=-3)=P(3X=-1.5)*P(-2Y=2)+P(3X=1.5)*P(-2Y=-2)=5/25*10/25+7/25*5/25=17/125\\
P(3X*(-2Y)=0)=P(3X=0)*P(-2Y=2)+P(3X=-1.5)*P(-2Y=0)+P(3X=0)*P(-2Y=0)+P(3X=1.5)*P(-2Y=0)+P(3X=0)*P(-2Y=-2)=89/125\\
P(3X*(-2Y)=3)=P(3X=1.5)*P(-2Y=2)+P(3X=-1.5)*P(-2Y=-2)=19/125\\
$
\\
\\
\begin{tabular}{|c|c|c|c|}
\hline
3X*(-2Y) & -3 & 0 & 3 \\
\hline
P: & 17/125 & 89/125 & 19/125  \\
\hline
\end{tabular}
\\
$
E(3X*(-2Y))=6/125\\
D(3X-2Y)=9*0.1184+4*0.56+6/125+6*0.04*(-0.2)=
1.0656+2.24+6/125-0.048=3.3056=33056/10000\\
$
За каким-то хером делим ответ на 16\\
$
33056/10000=2066/625\\
$
\newpage
\subsection{Задание 109}
\begin{tabular}{|c|c|c|c|}
\hline
X/Y & -2 & 0 & 1 \\
\hline
0: & 0.1 & 0.1 & 0.2  \\
\hline
1: & 0.3 & 0.1 & 0.2  \\
\hline
\end{tabular}
\\
Y*X:\\
-2 * 0 = 0\\
0 * 0 = 0\\
1 * 0 = 0\\
-2 * 1 = -2\\
0 * 1 = 0\\
1 * 1 = 1\\
\\
Для примера:\\
P(XY=0)=0.1+0.1+0.2+0.1=0.5\\
\\
\begin{tabular}{|c|c|c|c|}
\hline
X*Y & -2 & 0 & 1 \\
\hline
P: & 0.3 & 0.5 & 0.2  \\
\hline
\end{tabular}
\\
E(XY)=-0.6+0.2=-0.4\\
\\
\begin{tabular}{|c|c|c|}
\hline
X & 0 & 1  \\
\hline
P: & 0.4 & 0.6  \\
\hline
\end{tabular}
\\
\\
\begin{tabular}{|c|c|c|c|}
\hline
Y & -2 & 0 & 1 \\
\hline
P: & 0.4 & 0.2 & 0.4  \\
\hline
\end{tabular}
\\
$
EX=0.6\\
EX*X=0.6\\
EY=0.6\\
EY*Y=2\\
E^{2}(X)=0.16\\
E^{2}(Y)=0.36\\
DY=2-0.16=1.84\\
Cov(2X-3Y+5,Y-3X+2)=
$
\newpage
\subsection{Задание 110}
\begin{tabular}{|c|c|c|c|}
\hline
X & 0 & 1 & 2 \\
\hline
0: & 0.3 & 0.15 & 0.55  \\
\hline
\end{tabular}
\\
\\
\begin{tabular}{|c|c|c|c|}
\hline
Y & 0 & -1 & 3 \\
\hline
P: & 0.55 & 0.1 & 0.35  \\
\hline
\end{tabular}
\\
$
EX=15/100+2*55/100=1,25\\
EY=-0.1+3*35/100=0.95\\
E(X^2)=2.35\\
E(Y^2)=3.25\\
D(X)=2.35-1,5625=0.7875\\
D(Y)=3.25-0.9025=2.3475\\
X*Y:\\
0 * 0 = 0\\
1 * 0 = 0\\
2 * 0 = 0\\
0 * (-1) = 0\\
1 * (-1) = -1\\
2 * (-1) = -2\\
0 * 3 = 0\\
1 * 3 = 3\\
2 * 3 = 6\\
$
\\
\\
\begin{tabular}{|c|c|c|c|c|c|}
\hline
X*Y & -2 & -1 & 0 & 3 & 6 \\
\hline
P: & 0 & 0.1 & 0.65 & 0 & 0.25  \\
\hline
\end{tabular}
\\
E(XY)=-0.1+6*0.25=1.4\\
Cov(X,Y)=1.4-1.25*0.95=0.2125\\
E(2X-5Y-1)=2EX-5EY-1=2*1,25-5*0.95-1=-3.25\\
\end{document}