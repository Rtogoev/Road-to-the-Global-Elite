\documentclass[12pt]{article}
\usepackage[utf8]{inputenc} 
\usepackage[english, russian]{babel}
\usepackage{amsmath}
\usepackage{amsfonts}
\linespread{1.3}
\begin{document}
\begin{titlepage}
\begin{center}
\textbf{\huge Минорский}
\end{center}
\end{titlepage}
\tableofcontents
\newpage
\section{Введение в анализ}
\subsection{\S Свойства пределов. Раскрытие неопределенностей ввида $\frac{0}{0}$ и $\frac{\infty}{\infty}$}
\subsubsection{734}
1) тут сразу подставим
$$
\lim_{x\to 2} \frac{x^2-4x+1}{2x+1}= -3/5
$$
2) Вспоминаем формулы косинуса двойного угла
$$
\lim_{x\to \pi/4} \frac{1+sin2x}{1-cos4x}=
\lim_{x\to \pi/4} \frac{1+sin2x}{1-(1-2sin^{2}2x)}=
\lim_{x\to \pi/4} \frac{1+sin2x}{2sin^{2}2x)}=1
$$

\newpage
\subsubsection{735}
$$
\lim_{x\to 2} \frac{x^2-4}{x-2}=
\lim_{x\to 2} \frac{(x-2)(x+2)}{x-2}=4
$$

\newpage
\subsubsection{736}
Корни уравнения в знаменателе будут 2 и 1
$$
\lim_{x\to 2} \frac{x-2}{x^2-3x+2}=
\lim_{x\to 2} \frac{(x-2)}{(x-2)(x-1)}=1
$$

\newpage
\subsubsection{737}
$$
\lim_{x\to 3} \frac{x^2-9}{x^2-2x-3}=
\lim_{x\to 3} \frac{(x-3)(x+3)}{(x-3)(x+1)}=3/2
$$

\newpage
\subsubsection{738}
$$
\lim_{x\to \pi} \frac{tgx}{sin2x}=
\lim_{x\to \pi} \frac{sinx}{cosx*sin2x}=
\lim_{x\to \pi} \frac{sinx}{cosx*2sinx*cosx}=
\lim_{x\to \pi} \frac{1}{2cos^2 x}=1/2
$$

\newpage
\subsubsection{739}
$$
\lim_{x\to \pi/4} \frac{sinx-cosx}{cos2x}=
\lim_{x\to \pi/4} \frac{sinx-cosx}{cos^2 x- sin^2 x}=
\lim_{x\to \pi/4} \frac{sinx-cosx}{-(sinx-cosx)(cos x+ sin x)}=
-1/\sqrt{2}
$$

\newpage
\subsubsection{740}
$$
\lim_{x\to 0} \frac{x}{\sqrt{1+3x}-1}=
\lim_{x\to 0} \frac{x(\sqrt{1+3x}+1)}{(\sqrt{1+3x}-1)(\sqrt{1+3x}-1)}=
\lim_{x\to 0} \frac{x(\sqrt{1+3x}+1)}{1+3x-1}=
\lim_{x\to 0} \frac{\sqrt{1+3x}+1}{3}=2/3
$$

\newpage
\subsubsection{741}
$$
\lim_{x\to a} \frac{\sqrt{ax}-x}{x-a}=
\lim_{x\to a} \frac{(\sqrt{ax}-x)(\sqrt{ax}+x)}{(x-a)(\sqrt{ax}+x)}=
\lim_{x\to a} \frac{ax-x^2}{(x-a)(\sqrt{ax}+x)}=
$$
$$
=\lim_{x\to a} \frac{-x(x-a)}{(x-a)(\sqrt{ax}+x)}=\lim_{x\to a} \frac{-x}{\sqrt{ax}+x}=-1/2
$$


\newpage
\subsubsection{742}

$$
\lim_{x\to 1} \frac{\sqrt[3]{x}-1}{\sqrt{x}-1}=
$$
$$
x=t^6, t \to 1
$$
$$
\lim_{x\to 1} \frac{t^2-1}{t^3-1}=
\lim_{x\to 1} \frac{(t-1)(t+1)}{(t-1)(t^2+t+1)}=
\lim_{x\to 1} \frac{t+1}{t^2+t+1}=2/3
$$

\newpage
\subsubsection{743}

$$
\lim_{x\to 0} \frac{\sqrt[3]{1+mx}-1}{x}=
$$
$$
1+mx = t^3, t \to 1, x = \frac{t^3-1}{m}
$$
$$
\lim_{t\to 1} \frac{\sqrt[3]{t^3}-1}{\frac{t^3-1}{m}}=
\lim_{t\to 1} \frac{m(t-1)}{t^3-1}=
\lim_{t\to 1} \frac{m}{t^2+t+1}=m/3
$$

\newpage
\subsubsection{744}

$$
\lim_{x\to 0} \frac{\sqrt{1+x}-\sqrt{1-x}}{x}=
\lim_{x\to 0} \frac{(\sqrt{1+x}-\sqrt{1-x})(\sqrt{1+x}+\sqrt{1-x})}{x(\sqrt{1+x}+\sqrt{1-x})}=
$$
$$
\lim_{x\to 0} \frac{1+x-1+x}{\sqrt{1+x}+\sqrt{1-x}}=1
$$

\newpage
\subsubsection{745}

$$
\lim_{x\to \pi} \frac{\sqrt{1-tgx}-\sqrt{1+tgx}}{sin2x}=
\lim_{x\to \pi} \frac{(\sqrt{1-tgx}-\sqrt{1+tgx})(\sqrt{1-tgx}+\sqrt{1+tgx})}{sin2x(\sqrt{1-tgx}+\sqrt{1+tgx})}=
$$
$$
\lim_{x\to \pi} \frac{-2tgx}{sin2x(\sqrt{1-tgx}+\sqrt{1+tgx})}=
\lim_{x\to \pi} \frac{-2sinx}{cos*2sinx*cosx(\sqrt{1-tgx}+\sqrt{1+tgx})}=
$$
$$
=\lim_{x\to \pi} \frac{1}{-2cos^2 x}=-1/2
$$

\newpage
\subsubsection{746}

1)\\
$$
\lim_{x\to \infty} \frac{2x^2-1}{3x^2-4x}=
$$
$$
\lim_{x\to \infty} \frac{2x-1/x}{3x-4}=
$$
$$
\lim_{x\to \infty} \frac{2x}{3x}=2/3
$$

2)\\
$$
\lim_{x\to \infty} \frac{5x^3-7x}{1-2x^3}=
$$
$$
\lim_{x\to \infty} \frac{5x^2-7}{1/x-2x^2}=
$$\\
Относительно бесконечно больших, константы всегда принимаются за 0. Можно конечно вынести старшую степень,  тогда любое число, деленное на бесконечность, будет 0.
$$
\lim_{x\to \infty} \frac{5x^2}{2x^2}=5/2
$$

\newpage
\subsubsection{747}

1)\\
$$
\lim_{x\to \infty} \frac{3x-1}{x^2+1}=
$$
$$
\lim_{x\to \infty} \frac{3-1/x}{x+1/x}=
$$
$$
=\frac{3}{\infty} = 0
$$

\newpage
\subsubsection{748}

1)\\
$$
\lim_{x\to \infty} \frac{x^3-1}{x^2+1}=
$$
$$
\lim_{x\to \infty} \frac{x-1/x^2}{1+1/x^2}=
$$
$$
=\frac{\infty}{1} = \infty
$$

\newpage
\subsubsection{749}

1)\\
$$
\lim_{x\to \infty} \frac{\sqrt{x}-6x}{3x+1} ;
\sqrt{x} = a;
$$
$$
\lim_{x\to \infty} \frac{a-6a^2}{3a^2+1}=
$$
$$
=\lim_{x\to \infty} \frac{1-6a}{3a+1/a} = 
$$
$$
=\lim_{x\to \infty} \frac{-6a}{3a} = -2
$$

\newpage
\subsubsection{750}

1)\\
$$
\lim_{n\to \infty} \frac{3n}{1-2n}=
$$
$$
\lim_{n\to \infty} \frac{3}{1/n-2}=-3/2
$$


\newpage
\subsubsection{751}

1)\\
$$
\lim_{n\to \infty} \frac{\sqrt{2n^2+1}}{2n-1}=
$$
$$
\lim_{n\to \infty} \frac{\sqrt{2n^2+1}}{2n+1-2}=
$$
Отбросим константы, потому что они ни на что не влияют
$$
\lim_{n\to \infty} \frac{\sqrt{2n^2}}{2n}=
$$
$$
\lim_{n\to \infty} \frac{n\sqrt{2}}{2n}=\frac{\sqrt{2}}{2}
$$

\newpage
\subsection{\S Предел отношения $\frac{sin a}{a}$ при$  a \to 0$ }
\subsubsection{763}
$$
\lim_{x\to 0} \frac{sin4x}{x} = 
$$
$$
= \lim_{x\to 0} \frac{4sin4x}{4x} = 4 
$$

\newpage
\subsubsection{764}
$$
\lim_{x\to 0} \frac{sin(x/3)}{x} = 
$$
$$
= \lim_{x\to 0} \frac{(1/3)sin(x/3)}{x/3} = 1/3 
$$

\newpage
\subsubsection{765}
$$
\lim_{x\to 0} \frac{tgx}{x} = 
$$
$$
= \lim_{x\to 0} \frac{sinx}{xcosx} =  
$$
$$
= \lim_{x\to 0} \frac{1}{cosx} = 1 
$$

\newpage
\subsubsection{766}
$$
\lim_{x\to 0} \frac{sin^2(x/2)}{x^2} = 
$$
$$
\lim_{x\to 0} \frac{(1/4)sin^2(x/2)}{(1/4)x^2} =1/4 
$$

\newpage
\subsubsection{767}
$$
\lim_{x\to 0} \frac{1-cos2x}{xsinx} = 
$$
$$
\lim_{x\to 0} \frac{sin^2x+cos^2x-cos^2x+sin^2x}{xsinx} =  \lim_{x\to 0} \frac{2sinx}{x} = 2
$$

\newpage
\subsubsection{768}
$$
\lim_{x\to 0} \frac{sin3x}{\sqrt{x+2}-\sqrt{2}} = 
$$
$$
\lim_{x\to 0} \frac{sin3x(\sqrt{x+2}+\sqrt{2})}{x+2-2} = 6\sqrt{2}
$$

\newpage
\subsubsection{769}
$$
\lim_{h\to 0} \frac{sin(x+h)-sin(x-h)}{h} = 
$$
...
$$
sin(x+h)-sin(x-h) = 2sin\frac{(x+h-x+h)}{2}*cos\frac{x+h+x-h}{2} = 2sin(2h/2)*cos(2x/2)
$$
...
$$
\lim_{h\to 0} \frac{2sin(2h/2)*cos(2x/2)}{h} = 2cosx
$$

\newpage
\subsubsection{770}
1)
$$
\lim_{x\to 0} \frac{arctgx}{x} = 
$$
...
$$
arctgx = y, x = tgy, y \to  0 
$$
...
$$
\lim_{y\to 0} \frac{y}{tgy} = \lim_{y\to 0} \frac{ycosy}{siny} =\lim_{y\to 0}  cosy = 1 
$$

ниже ошибка\\
 \\
2)
$$
\lim_{x\to 1/2} \frac{arcsin(1-2x)}{4x^2-1} = 
$$
...
$$
arcsin(1-2x) = y, x = siny, y \to  0 
$$
...
$$
\lim_{y\to 0} \frac{y}{4sin^2y-1} = \lim_{y\to 0} \frac{y}{4sin^2y-1} =\lim_{y\to 0} \frac{1}{4-1/sin^2y}=\lim_{y\to 0} (1/4 - sin^2y) = 1/4
$$

\newpage
\subsubsection{771}
1)
$$
\lim_{x\to 0} \frac{1-cosx}{x^2} = 
\lim_{x\to 0} \frac{1-cos^2x}{x^2(1+cosx)}=
$$
$$
=\lim_{x\to 0} \frac{1}{1+cosx}=1/2
$$

\end{document}
