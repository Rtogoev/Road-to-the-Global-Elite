\documentclass[12pt]{article}
\usepackage[utf8]{inputenc} 
\usepackage[english, russian]{babel}
\linespread{1.5}
\begin{document}
\begin{titlepage}
\begin{center}
\textbf{\huge Теория  и решение примеров Шага 4, Ступени 2}
\end{center}
\end{titlepage}
\tableofcontents
\newpage
\section{\S 20. Комплексные числа в алгебраической форме}
\begin{center}\textbf{Теория}\end{center}
$ (a+bi) + (c+di) = (a+c)+(b+d)i; $ \\
$ (a+bi)(c+di) = (ac-bd)+(ad+bc)i; $\\
$ z_1 = a_1+b_1i;$\\
$ z_2 = a_2+b_2i;$\\
$ z_1 - z_2 = (a_1-a_2)+(b_1-b_2)i; $\\
$
\displaystyle
\frac{z_1}{z_2}=
\frac{a_1+b_1i}{a_2+b_2i}=
\frac{a_1 a_2 + b_1 b_2}{a_2^2+b_2^2}+
\frac{b_1 a_2 - a_1 b_2}{a_2^2+b_2^2}i
$
\subsection{Задание 20.1}

\textbf{a)}\\
\begin{center}\textbf{Пример}\end{center}
$$(2+i)(3-i)+(2+3i)(3+4i) = ?;$$
\begin{center}\textbf{Решение}\end{center}
$(2+i)(3-i)=(2\cdot3+1)+(-2+3)i = 7+i;$\\
$(2+3i)(3+4i) =(2\cdot3-3\cdot4)+( 2\cdot 4+3\cdot3)i=(6-12)+(8+9)i=-6+17i;$\\
$(7+i)+(-6+17i)=(7-6)+(1+17)i=1+18i;$

\newpage
\textbf{б)}\\
\begin{center}\textbf{Пример}\end{center}
$$(2+i)(3+7i)-(1+2i)(5+3i) = ?;$$
\begin{center}\textbf{Решение}\end{center}
$(2+i)(3+7i) = (2\cdot3-7)+(2\cdot7+3)i=-1+17i;$\\
$(1+2i)(5+3i) =(5-6)+(3+10)i = -1+13i;$\\
$(-1+17i)-(-1+13i)=(-1+17i)+(1-13i)=4i;$

\newpage
\textbf{в)}\\
\begin{center}\textbf{Пример}\end{center}
$$(4+i)(5+3i)-(3+i)(3-i) = ?;$$
\begin{center}\textbf{Решение}\end{center}
Используя формулу сокращенного умножения:
$$(a+b)(a-b)=a^2-b^2;$$
$(3+i)(3-i) = 9 + 1 = 10;$\\
$(4+i)(5+3i)= (4\cdot5 - 3)+(4\cdot3+5)i=17+17i;$\\
$17+17i-10=7+17i;$

\newpage
\textbf{г)}\\
\begin{center}\textbf{Пример}\end{center}
$$
\frac{(5+i)(7-6i)i}{3+i} = ?;
$$
\begin{center}\textbf{Решение}\end{center}
$(5+i)(7-6i)=(5*7+6)+(-5*6+7)i=41-23i;$\\
$
\displaystyle
\frac{41-23i}{3+i}=
\frac{41\cdot3-23}{10}+\frac{-23\cdot3-41}{10}i=
\frac{100}{10}+\frac{-110}{10}i=
10-11i;
$

\newpage

\textbf{д)}\\
\begin{center}\textbf{Пример}\end{center}
$$
\frac{(5+i)(3+5i)}{2i}=?;
$$
\begin{center}\textbf{Решение}\end{center}
$(5+i)(3+5i)=(5\cdot3-5)+(5\cdot5 + 3)i=10+28i;$\\
\\
В ситуации, когда действительная часть в знаменателе равна нулю,
не нужно использовать формулу деления комплексных чисел.
Расписать все выражение, как сумму дробей и вычислить получившиеся дроби.\\
\\
$
\displaystyle
\frac{10+28i}{2i}=
\frac{10}{2i}+
\frac{28i}{2i}=...
$\\
\\
Тут домножили и разделили на i мнимую часть\\
\\
$ 
\displaystyle
...=14 + \frac{5i}{i^2} = 14-5i;
$\\
\\
с помощью форулы деления комплексных чисел\\
\\
$
\displaystyle
\frac{10+28i}{2i}=
\frac{56}{4}+
\frac{-20}{4}i=
14-5i;
$
\end{document}