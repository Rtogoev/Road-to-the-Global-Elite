\documentclass[12pt]{article}
\usepackage[utf8]{inputenc} 
\usepackage[english, russian]{babel}
\usepackage{amsmath}
\usepackage{amsfonts}
\linespread{1.3}
\begin{document}
\begin{titlepage}
\begin{center}
\textbf{\huge Теория  и решение примеров Шага 4, Ступени 2}
\end{center}
\end{titlepage}
\tableofcontents
\newpage
\section{\S 20. Комплексные числа в алгебраической форме}
\begin{center}\textbf{Теория}\end{center}
$ (a+bi) + (c+di) = (a+c)+(b+d)i; $ \\
$ (a+bi)(c+di) = (ac-bd)+(ad+bc)i; $\\
$ z_1 = a_1+b_1i;$\\
$ z_2 = a_2+b_2i;$\\
$ z_1 - z_2 = (a_1-a_2)+(b_1-b_2)i; $\\
$
\displaystyle
\frac{z_1}{z_2}=
\frac{a_1+b_1i}{a_2+b_2i}=
\frac{a_1 a_2 + b_1 b_2}{a_2^2+b_2^2}+
\frac{b_1 a_2 - a_1 b_2}{a_2^2+b_2^2}i
$
\subsection{Задание 20.1}

\textbf{a)}\\
\begin{center}\textbf{Пример}\end{center}
$$(2+i)(3-i)+(2+3i)(3+4i) = ?;$$
\begin{center}\textbf{Решение}\end{center}
$(2+i)(3-i)=(2\cdot3+1)+(-2+3)i = 7+i;$\\
$(2+3i)(3+4i) =(2\cdot3-3\cdot4)+( 2\cdot 4+3\cdot3)i=(6-12)+(8+9)i=-6+17i;$\\
$(7+i)+(-6+17i)=(7-6)+(1+17)i=1+18i;$

\newpage
\textbf{б)}\\
\begin{center}\textbf{Пример}\end{center}
$$(2+i)(3+7i)-(1+2i)(5+3i) = ?;$$
\begin{center}\textbf{Решение}\end{center}
$(2+i)(3+7i) = (2\cdot3-7)+(2\cdot7+3)i=-1+17i;$\\
$(1+2i)(5+3i) =(5-6)+(3+10)i = -1+13i;$\\
$(-1+17i)-(-1+13i)=(-1+17i)+(1-13i)=4i;$

\newpage
\textbf{в)}\\
\begin{center}\textbf{Пример}\end{center}
$$(4+i)(5+3i)-(3+i)(3-i) = ?;$$
\begin{center}\textbf{Решение}\end{center}
Используя формулу сокращенного умножения:
$$(a+b)(a-b)=a^2-b^2;$$
$(3+i)(3-i) = 9 + 1 = 10;$\\
$(4+i)(5+3i)= (4\cdot5 - 3)+(4\cdot3+5)i=17+17i;$\\
$17+17i-10=7+17i;$

\newpage
\textbf{г)}\\
\begin{center}\textbf{Пример}\end{center}
$$
\frac{(5+i)(7-6i)i}{3+i} = ?;
$$
\begin{center}\textbf{Решение}\end{center}
$(5+i)(7-6i)=(5*7+6)+(-5*6+7)i=41-23i;$\\
\\
$
\displaystyle
\frac{41-23i}{3+i}=
\frac{41\cdot3-23}{10}+\frac{-23\cdot3-41}{10}i=
\frac{100}{10}+\frac{-110}{10}i=
10-11i;
$

\newpage

\textbf{д)}\\
\begin{center}\textbf{Пример}\end{center}
$$
\frac{(5+i)(3+5i)}{2i}=?;
$$
\begin{center}\textbf{Решение}\end{center}
$(5+i)(3+5i)=(5\cdot3-5)+(5\cdot5 + 3)i=10+28i;$\\
\\
В ситуации, когда действительная часть в знаменателе равна нулю,
не нужно использовать формулу деления комплексных чисел.
Расписать все выражение, как сумму дробей и вычислить получившиеся дроби.\\
\\
$
\displaystyle
\frac{10+28i}{2i}=
\frac{10}{2i}+
\frac{28i}{2i}=...
$\\
\\
Тут домножили и разделили на i мнимую часть\\
\\
$ 
\displaystyle
...=14 + \frac{5i}{i^2} = 14-5i;
$\\
\\
с помощью форулы деления комплексных чисел\\
\\
$
\displaystyle
\frac{10+28i}{2i}=
\frac{56}{4}+
\frac{-20}{4}i=
14-5i;
$


\newpage
\textbf{е)}\\
\begin{center}\textbf{Пример}\end{center}
$
\displaystyle
\frac{(1+3i)(8-i)}{(2+i)^2}=?;
$\\
\begin{center}\textbf{Решение}\end{center}
$
(1+3i)(8-i)=(8+3)+(-1+24)i=11+23i;
$\\
$
(2+i)^2=(2+i)\cdot(2+i)=(2\cdot2-1)+(2+2)i=3+4i;
$\\
\\
$
\displaystyle
\frac{11+23i}{3+4i}=
\frac{11\cdot3+23\cdot4}{9+16}+
\frac{23\cdot3-11\cdot4}{9+16}i=
\frac{33+92}{25}+
\frac{69-44}{25}=\\
=\frac{125}{25}+
\frac{25}{25}i=5+i;
$\\

\newpage
\textbf{ж)}\\
\begin{center}\textbf{Пример}\end{center}
$
\displaystyle
\frac{(2+i)(4+i)}{1+i}=?;
$\\
\begin{center}\textbf{Решение}\end{center}
$
(2+i)(4+i)=(8-1)+(2+4)i=7+6i;
$\\
\\
$
\displaystyle
\frac{7+6i}{1+i}=
\frac{7+6}{2}+
\frac{6-7}{2}i=
\frac{13}{2}-
\frac{1}{2}i;
$\\

\newpage
\textbf{з)}\\
\begin{center}\textbf{Пример}\end{center}
$
\displaystyle
\frac{(3-i)(1-4i)}{2-i}=?;
$\\
\begin{center}\textbf{Решение}\end{center}
$
(3-i)(1-4i)=(3-4)+(-12-1)i=-1-13i;
$\\
\\
$
\displaystyle
\frac{-1-13i;}{2-i}=
\frac{-2+13}{5}+
\frac{-26-1}{5}i=
\frac{11}{5}+
\frac{27}{5}i;
$

\newpage
\textbf{и)}\\
\begin{center}\textbf{Пример}\end{center}
$
(2+i)^3+(2-i)^3=?;
$\\
\begin{center}\textbf{Решение}\end{center}
по формуле суммы куббов:
$$
a^3+b^3=(a+b)(a^2-ab+b^2)
$$
$
((2+i)+(2-i))
(
(2+i)^2-(2-i)(2+i)+(2-i)^2
)=
4\cdot((2+i)^2-5+(2-i)^2);
$\\
\\
по формуле квадрата суммы и разности:
$$(a+b)^2=(a^2+2ab+b^2)$$
$$(a-b)^2=(a^2-2ab+b^2)$$
$
(2+i)^2 = 4+4i-1;\\
(2-i)^2 = 4-4i-1;\\
4\cdot(4+4i-1-5+4-4i-1)=4\cdot(4+4-1-5-1)=4(8-7)=4;
$\\

\newpage
\textbf{к)}\\
\begin{center}\textbf{Пример}\end{center}
$
(3+i)^3-(3-i)^3=?;
$\\
\begin{center}\textbf{Решение}\end{center}
по формуле разности куббов:
$$
a^3-b^3=(a-b)(a^2+ab+b^2)
$$
$
((3+i)(3-i))
((3+i)^2+(3+i)(3-i)+(3-i)^2)=\\
=2i(9+6i-1+9-3i+3i+1+9-6i-1)=
2i(9-1+9+1+9-1)=
2i\cdot26=
52i
$

\newpage
\textbf{л)}\\
\begin{center}\textbf{Пример}\end{center}
$
\displaystyle
\frac{(1+i)^5}{(1-i)^3}=?
$\\
\begin{center}\textbf{Решение}\end{center}
$
\displaystyle
\frac{(1+i)^5}{(1-i)^3}=
\frac{(1+i)^2(1+i)^2(1+i)}{(1-i)^2(1-i)}=...
\\
\\
(1+i)^2 = 1+2i-1=2i\\
(1-i)^2 = 1-2i-1=-2i\\
\\
...=\frac{2i\cdot2i}{-2i(1-i)}=
\frac{2i(1+i)}{i-1}=
\frac{2i-2}{i-1}=
\frac{2(i-1)}{(i-1)}=2
$

\newpage
\textbf{м)}\\
\begin{center}\textbf{Пример}\end{center}
$
\displaystyle
(-\frac{1}{2} +\frac{\sqrt3}{2}i)^3=?
$\\
\begin{center}\textbf{Решение}\end{center}
$
\displaystyle
(-\frac{1}{2} +\frac{\sqrt3}{2}i)^3=
\frac{(-1+\sqrt3i)^2(-1+\sqrt3i)}{8}=...\\
\\
(-1+\sqrt3i)^2=1-2\sqrt3i-3=-2\sqrt3i-2\\
\\
...=\frac{(-2\sqrt3i-2)(-1+\sqrt3i)}{8}=
\frac{2\sqrt3i+6+2-2\sqrt3i}{8}=
\frac{8}{8}=1
$

\newpage
\subsection{Задание 20.2}

\begin{center}\textbf{Пример}\end{center}
Вычислить: 
$
i^{77}, i^{98}, i^{-57}, i^n,
$ n  - целое число
\begin{center}\textbf{Решение}\end{center}
$
i^2=-1\\
i^4=1\\
i^{77}=i^{76}i=(i^4)^{19}i=1^{19}i=i\\
i^{98}=(i^4)^{24}i^2=i^2=-1\\
$
\begin{equation*}
i^n= 
 \begin{cases}
n = 4k,n \in \mathbb {Z}, i^n = 1\\
n = 4k+1,n \in \mathbb {Z}, i^n = i\\
n = 4k+2,n \in \mathbb {Z}, i^n = -1\\
n = 4k+3,n \in \mathbb {Z}, i^n = -i\\   
 \end{cases}
\end{equation*}

\newpage
\subsection{Задание 20.3}
Доказать равенство... или сделать так, чтобы левое стало равно правому\\
\\
\textbf{a)}\\
\begin{center}\textbf{Пример}\end{center}
$
(1+i)^8n=2^4n
$
\begin{center}\textbf{Решение}\end{center}
Сводим к одной степени:
$$
(1+i)^2)^{4n}=2^{4n}\\
$$
Что-то уже можно посчитать:
$$
(1+i)^2 = 2i
$$
$
(2i)^{4n}=2^{4n}\\
2^{4n}i^{4n}=2^{4n}\\
2^{4n}i=2^{4n} \\
$
Что-то уже можно посчитать:
$$
i^4=i^2\cdot i^2=-1\cdot(-1)=1
$$
$
2^{4n}=2^{4n}
$

\newpage
\textbf{б)}\\
\begin{center}\textbf{Пример}\end{center}
$
(1+i)^{4n}=(-1)2^{2n}
$
\begin{center}\textbf{Решение}\end{center}
$
(((1+i)^2)^2)^n=-2^{2n}\\
((2i)^2)^n=-4^n\\
-4^n=-4^n
$

\newpage
\subsection{Задание 20.4}
Решить систему уравнений:\\
во всех заданиях очень помогал финт ушами про избавление от иррациональности в знаменателе.\\
\textbf{a)}\\
\begin{center}\textbf{Пример}\end{center}
\begin{equation*}
 \begin{cases}
(1+i)z_1+(1-i)z_2=1+i\\
(1-i)z_1+(1+i)z_2=1+3i
 \end{cases}
\end{equation*}
\begin{center}\textbf{Решение}\end{center}
Во втором уравнении вынесем все, кроме $z_1$ в правую часть:\\
\\
$
\displaystyle
z_1=\frac{1+i-(1-i)z_2}{1+i}
=1-\frac{(1-i)z_2}{1+i}
$\\
\\
Подставим в первое уравнение:\\
\\
$
\displaystyle
(1-i)(1-\frac{(1-i)z_2}{1+i})+(1+i)z_2=1+3i\\
1-i-\frac{(1-i)^2\cdot z_2}{1+i}+(1+i)z_2=1+3i\\
...\\
(1-i)^2=-2i\\
...\\
1-i+\frac{2iz_2}{1+i}+z_2+z_2i=1+3i\\
\frac{2iz_2}{1+i}+z_2+z_2i=1+3i-1+i\\
z_2(\frac{2i}{1+i}+1+i)=4i\\
z_2=\frac{4i}{\frac{2i}{1+i}+1+i}\\
...\\
(1+i)^2=2i\\
\frac{2i}{1+i}+1+i=
\frac{2i+2i}{1+i}=\frac{4i}{1+i}\\
...\\
z_2=\frac{4i}{\frac{4i}{1+i}}=4i\frac{1+i}{4i}=1+i\\
$\\
Подставим в выведенное ранее $z_1$:\\
$\\
z_1=1-\frac{(1-i)(1+i)}{1+i}=1+i-1=i
$

\newpage
\textbf{б)}\\
\begin{center}\textbf{Пример}\end{center}
\begin{equation*}
 \begin{cases}
iz_1+(1+i)z_2=2+2i\\
2iz_1+(3+2i)z_2=5+3i\\
 \end{cases}
\end{equation*}
\begin{center}\textbf{Решение}\end{center}
Во втором уравнении выразим $z_1$:\\
$\\
\displaystyle
z_1=\frac{5+3i-(3+2i)z_2}{2i}=\\
$\\
Подставим в первое уравнение:\\
\\
$
\displaystyle
i\frac{5+3i-(3+2i)z_2}{2i}+z_2+iz_2=2(1+i)\\
\frac{5+3i-(3+2i)z_2}{2}+z_2+iz_2=2(1+i)\\
\frac{5+3i}{2}-\frac{(3+2i)z_2}{2}+z_2+iz_2=2(1+i)\\
-\frac{(3+2i)z_2}{2}+z_2+iz_2=2(1+i)-\frac{5+3i}{2}\\
\frac{-3z_2-2iz_2+2z_2+2iz_2}{2}=2(1+i)-\frac{5+3i}{2}\\
-\frac{z_2}{2}=2(1+i)-\frac{5+3i}{2}\\
-z_2=4(1+i)-5-3i=4+4i-5-3i\\
z_2=-4-4i+5+3i=1-i\\
$\\
Подставим полученное значение $z_2$ в формулу $z_1$:\\
\\
$
\displaystyle
z_1=\frac{5+3i-(3+2i)(1-i)}{2i}=
\frac{5+3i-3+3i-2i-2}{2i}=\frac{4i}{2i}=2\\
$

\end{document}